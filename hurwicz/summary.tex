\documentclass[12pt]{article}

\usepackage[top=0.8in,bottom=1.1in,left=1.1in,right=1.1in]{geometry}
\usepackage{enumitem}
\usepackage{xcolor}
\usepackage{tcolorbox}
\usepackage{titlesec}


\setlist[itemize]{leftmargin=1.8em,rightmargin=1.8em}
\titlespacing*{\section}{0pt}{1.4ex}{1.6ex}
\titleformat{\section}
  {\normalsize}
  {}
  {0pt}
  {\underline}

\begin{document}

\begin{center}
{\large{\textbf{Incentive Compatibility under Informational Decentralization}}}
\end{center}

\vspace{0.25em}

\section*{Background}

\noindent
The paper takes the classical impossibility results of mechanism design as given. It does not:

\begin{itemize}
    \item weaken or refute the Revelation Principle
    \item overturn any impossibility results within their assumptions
    \item rely on repeated-game discipline, reputation effects, or exogenous commitment devices
\end{itemize}


\section*{Summary of Paper}

\noindent
A subclass of \emph{non revelation-equivalent indirect mechanisms} exists which permits incentive compatibility under conditions of informational decentralization. The existence of this class is anticipated in theory. The class:

\begin{itemize}
    \item falls into the non–revelation-equivalent class under standard theory
    \item employs multidimensional, non-scalar message spaces 
    \item has compartmentalized layers running in parallel
    \item uses a dual enforcement regime to price deviations in unobserved strategic dimensions
    \item ensures all deviations that alter equilibrium allocations alter continuation value
\end{itemize}


\section*{Implications for Economics and Computer Science}

The mechanism lies outside the direct-mechanism paradigm that underlies much distributed systems theory. Its existence clarifies the scope of several canonical results in computer science by isolating the modelling assumptions on which they depend.

\begin{itemize}
    \item Incentive compatibility can arise from pricing non-verifiable dimensions via uncertainty. (econ)
    \item Direct-mechanism models are excluded from implementing this solution by construction. (econ)
    \item Some strategic dimensions must remain private to sustain enforcement. (econ)
    \item Several canonical impossibility results in computer science are conditional on direct mechanism modeling assumptions. (cs)
    \item In fee mechanisms, UIC + MIC become possible at the cost of ex post verifiability (cs)
\end{itemize}

\end{document}

