\documentclass[oneside]{article}   	% use "amsart" instead of "article" for AMSLaTeX format
\usepackage{geometry}                		% See geometry.pdf to learn the layout options. There are lots.
\geometry{letterpaper}                   		% ... or a4paper or a5paper or ... 
\geometry{legalpaper, portrait, margin=1in}
\usepackage[parfill]{parskip}    			% Activate to begin paragraphs with an empty line rather than an indent
\usepackage{graphicx}				% Use pdf, png, jpg, or eps§ with pdflatex; use eps in DVI mode
\usepackage{amssymb}
\usepackage{multicol}
\usepackage{abstract} 
\usepackage{graphicx}
\usepackage{caption}
\usepackage{changepage}
\usepackage{hyperref}

\usepackage{array}
\usepackage{multirow}
\usepackage{amssymb}
\usepackage{gensymb}
\usepackage{tabularx}
\usepackage{extarrows}
\usepackage{booktabs}
\usepackage{cite}

\usepackage[bottom]{footmisc}

\title{Modifying Utility Function to Include Heterogenous Goods}
\author{
  David Lancashire\\
  \texttt{david.lancashire@gmail.com}\\
}
\begin{document}
\maketitle

\subsection*{1.1 Modifying the Utility Function}
\vspace{0.5em}

Our existing equation for utility is:

\[
u_j^U(\theta_j, \{\theta_{j'}\}_{j' \in \mathcal{U} \setminus \{j\}}; p_j) := 
\begin{cases} 
\theta_j - p_j - \delta_t & \text{if } j \in S, \\
0 & \text{otherwise.}
\end{cases}
\]
\vspace{0.5em}

We have two kinds of utility:

\begin{itemize}
  \item \textbf{public good:} the cost-of-attack provided by the security function.
  \item \textbf{private good:} any additional benefits offered in exchange for fee flow.
\end{itemize}

We have two kinds of fees:

\begin{itemize}
  \item \textbf{public fee:} any fee allocated to the public good
  \item \textbf{private fee:} any fee allocated to a private good
\end{itemize}

The price paid by user j is the sum of their public and private fees.

$$
p_j = p_{pub}^j + p_{priv}^j
$$

Their valuation $\theta_j$ to the sum of their public and private utility functions:

$$
\theta_j = U_{pub}^j + U_{priv}^j
$$

The valuation of the public good is a monotonically-increasing function of all public fees in the block.

$$
U_{pub}^j = f_{pub}^j\left(\sum_{k \in S} p_{pub}^{k}\right)
$$

The valuation of the private good is a monotonically-increasing function of the users's own private fees.

$$
U_{priv}^j = f_{priv}^j(p_{priv}^j)
$$

This gives us a revised valuation equation:

$$
\theta_j = f_{pub}^j\left(\sum_{k \in S} p_{pub}^{k}\right) + f_{priv}^j(p_{priv}^j)
$$

As users do not face a variable "burn" for transaction inclusion, we remove it from our equation

\[
u_j^U\left(...\right) : =
\begin{cases}
\left(
	f_{pub}^j
		\left(\sum_{k \in S} p_{pub}^{k}\right) 
		+ f_{priv}^j(p_{priv}^j) 
\right)  -   \left(p_{pub} + p_{priv}\right) & \text{if } j \in S, \\ 0 & \text{otherwise.}
\end{cases}
\]

\subsection*{Utility Equation and Free-Riding}

Users reduce their public fee by $p_{fr}$ and replace it with a private payment used to allocate a different good (including cash discounts, refunds). Under conditions where the diminished competition that results permits the producer to redirect fees away from the security-function, any gain in utility can be split between the user and the producer. We can model this as the redirection of a portion of our fee into a second utility function.

$$
p_j = \left( p_{pub}^j - p_{fr}^j \right) + \left( p_{priv}^j + p_{fr}^j \right)
$$

This gives us the following valuation function:

$$
\theta_j = f_{pub}^j\left(\sum_{k \in S} p_{pub}^{k} - p_{fr}^j \right) + f_{priv}^j( p_{priv}^j + p_{fr}^j )
$$

And the following utility function:

$$
u_j^U\left(...\right) : =
\theta_j = f_{pub}^j\left(\sum_{k \in S} p_{pub}^{k} - p_{fr}^j \right) + f_{priv}^j( p_{priv}^j + p_{fr}^j )  -  \left(p_{pub} + p_{priv}\right)
$$

Total fees are unchanged, so free-riding is rational if the re-allocation increases utility, i.e.:

\[
f_{pub}^j\left(\sum_{k \in S} p_{pub}^{k} - p_{fr}^j \right) + f_{priv}^j( p_{priv}^j + p_{fr}^j )
> 
f_{pub}^j\left(\sum_{k \in S} p_{pub}^{k}\right) + f_{priv}^j(p_{priv}^j)
\]


\subsection*{1.3. Eliminating Free-Riding}
\vspace{0.5em}

The free-rider problem exploits the same vulnerability that \textit{private collusion} does. As per Samuelson (1954), \textit{private collusion} becomes suboptimal at the \textit{utility possibilities frontier} because the marginal utility of all goods is in equilibrium. This is why our social choice rule has to be pareto optimality.

\LARGE
\begin{adjustwidth}{1.5em}{1.5em}
\begin{math}
\frac{u_{pub}^i}{u_{priv}^i} = \frac{F_{pub}}{F_{priv}}
\end{math}
\end{adjustwidth}
\normalsize

Free-riding remains rational at the \textit{utility possibilities frontier}!

The reason is that even if the marginal utility of all available goods is in equilibrium and we all truthfully reveal our preference for that outcome, participants in that equilibrium can still "underpay" for the form of utility offered by the public good at that level and increase their individual welfare, dropping the amount of utility produced at the levels users have declared socially and individually optimal. As per the Revelation Principle, if the new level of public good provision is indeed individually optimal, the user was lying when they revealed their preference for the former equilibrium.


\pagebreak
\end{document}

