%%%%%%%%%%%%%%%%%%%%%%%%%%%%%%%%%%%%%%%%%%%%%%%%%%%%%%%%%%%%%%%%%%%%%%%%

%%% LaTeX Template for AAMAS-2025 (based on sample-sigconf.tex)
%%% Prepared by the AAMAS-2025 Program Chairs based on the version from AAMAS-2025. 

%%%%%%%%%%%%%%%%%%%%%%%%%%%%%%%%%%%%%%%%%%%%%%%%%%%%%%%%%%%%%%%%%%%%%%%%

%%% Start your document with the \documentclass command.


%%% == IMPORTANT ==
%%% Use the first variant below for the final paper (including auithor information).
%%% Use the second variant below to anonymize your submission (no authoir information shown).
%%% For further information on anonymity and double-blind reviewing, 
%%% please consult the call for paper information
%%% https://aamas2025.org/index.php/conference/calls/submission-instructions-main-technical-track/

%%%% For anonymized submission, use this
\documentclass[sigconf,anonymous]{aamas} 

%%%% For camera-ready, use this
%\documentclass[sigconf]{aamas} 


%%% Load required packages here (note that many are included already).

\usepackage{balance} % for balancing columns on the final page
\usepackage{xcolor}
\usepackage{soul}
\usepackage{enumitem}
\usepackage{bm}
\usepackage{mathtools}
\usepackage{multicol}
\usepackage{multirow}
\usepackage{array} % for array environments
\usepackage{pifont}% http://ctan.org/pkg/pifont
\newcommand{\cmark}{\ding{51}}%
\newcommand{\xmark}{\ding{55}}%

\newcommand{\sankarshan}[1]{\textcolor{magenta}{#1}}
\newcommand{\manisha}[1]{\textcolor{cyan}{#1}}
\newcommand{\varul}[1]{\textcolor{red}{#1}}

\setlength{\fboxsep}{10pt}  % Padding inside the box
\setlength{\fboxrule}{1pt}  % Border thickness


\usepackage{tikz}
\usetikzlibrary{arrows.meta, positioning}


%%%%%%%%%%%%%%%%%%%%%%%%%%%%%%%%%%%%%%%%%%%%%%%%%%%%%%%%%%%%%%%%%%%%%%%%

%%% AAMAS-2025 copyright block (do not change!)

%%% \setcopyright{ifaamas}
%%% \acmConference[AAMAS '25]{Proc.\@ of the 24th International Conference
%%% on Autonomous Agents and Multiagent Systems (AAMAS 2025)}{May 19 -- 23, 2025}
%%% {Detroit, Michigan, USA}{A.~El~Fallah~Seghrouchni, Y.~Vorobeychik, S.~Das, A.~Nowe (eds.)}
%%% \copyrightyear{2025}
%%% \acmYear{2025}
%%% \acmDOI{}
%%% \acmPrice{}
%%% \acmISBN{}


%%%%%%%%%%%%%%%%%%%%%%%%%%%%%%%%%%%%%%%%%%%%%%%%%%%%%%%%%%%%%%%%%%%%%%%%

%%% == IMPORTANT ==
%%% Use this command to specify your EasyChair submission number.
%%% In anonymous mode, it will be printed on the first page.

\acmSubmissionID{1330}
\newtheorem{Claim}{Claim}
\newtheorem{Remark}{Remark}
\newtheorem{example}{Example}
\newtheorem{assumption}{Assumption}

% Routing Threshold-based Randomized Transaction Fee Mechanism
\newcommand{\ourTFM}{\textsc{RTR-TFM}}
\renewcommand{\paragraph}[1]{\smallskip\noindent\textbf{#1}}
\newcommand{\ourOPT}{\texttt{OPT$_{\texttt{RTR}}$}}

\title[Routing Threshold-based Randomized Transaction Fee Mechanism]{RTR-TFM: A Routing Threshold-based Randomized Transaction Fee Mechanism}

\author{David Lancashire}
\affiliation{
  \institution{Proclus Technologies}
  \city{Bangkok}
  \country{Thailand}}
\email{david@saito.tech}

\begin{abstract}
Transaction Fee Mechanism (TFM) design explores the strategic interaction between users and block producers in permissionless networks. Classic TFMs give block producers control over the transactions they include in blocks, creating problems preventing collusion and achieving a socially-optimal fee equilibrium. Given previous research showing existing TFMs only solve these problems under strictly-limited conditions, we introduce RTR-TFM, which achieves collusion-resilience and incentive compatibility through a novel "routing-based" mechanism that derives the right to chain-extension and network payouts separately from the measured efficiency competing nodes demonstrate with the collection and sharing of transactions. This paper introduces the mechanism and demonstrates its robustness to the collective action problems that prevent incentive compatibility in other TFMs.
/end{abstract}

\keywords{Transaction Fee Mechanism, Leonid Hurwicz, Incentive Compatibility, Free-Riding, Collective Action Problems, Blockchain, Censorship Resistance, Collusion-Resilience, Distributed Systems}


\begin{document}
\pagestyle{fancy}
\fancyhead{}

\maketitle 



\section{Introduction \label{sec::introduction}}

\emph{Transaction Fee Mechanisms} (TFMs) refer to a class of distributed systems in which a consensus mechanism governs the allocation of the same resource that it uses to incentivize its own provision. Unlike traditional mechanisms, where the number of honest and dishonest processes is static, in TFMs voting power is dynamic — it adjusts with the payouts issued by the mechanism. This introduces the ability for attacks intended to extract profits from this circular flow of funds to compromise the stability of the mechanism.

As more and more of these attacks have been identified in practice, computer scientists have typically named them after the "mechanism-specific" techniques they exploit, resulting in a wide array of terminology such as sybil attacks, block-orphaning attacks, selfish mining attacks, fee manipulation attacks, eclipse attacks, side-contract payments, and others. While most researchers treat these vulnerabilities as isolated technical challenges, a few scholars have applied concepts from mechanism design and economics to ask whether general solutions are possible. Unfortunately, this has led to a series of impossibility results that suggest socially optimal TFMs may be infeasible.

This paper challenges these results by identifying the exact equilibrium in which all such attacks are irrational. It argues that three distinct types of goal conflict -— self-interest, free-riding, and strategic manipulation —- are what prevent this equilibrium from being implemented by most \textit{TFMs}. We then review the earlier work in the field to show why they believe the problem is insolvable: their conclusions result from a reliance on auction models which limits their ability to address all three types of goal conflict or even handle the informational complexity necessary to compute the required equilibria.

Using the language of mechanism design, this paper demonstrates that the social choice rule needed to achieve fee-optimality and collusion-resilience is \textit{pareto optimality}, but the direct mechanisms used to model TFMs are incapable of implementing this rule, as doing so requires multi-dimensional preference revelation across a high-dimensional preference space -- a level of informational complexity that composable algorithms cannot handle. While Maskin's Revelation Principle teaches that a direct mechanism must exist for any indirect mechanism, in this case achieving optimality requires decomposable algorithms that use the "no-trade option" to reduce the complexity of computation and limit the scope of the state transitions proposed to those consistent with an efficiency shift towards \textit{paretop optimality}.

Since familiarity with economics is needed to understand what our three types of goal conflict are and why they exist, the next section of this paper begins by identifying the novel characteristics of TFMs, and showing how they create problems with self-interest, free-riding and strategic manipulation. We then show why \textit{pareto optimality} is the social choice rule needed to eliminate all three, which leads to a review of the impossibility results mentioned above and a discussion of how the conclusions of these papers reflect the informational limitations of the auction models they use to analyse the problem.

In the second half of this paper, we introduce a novel class of indirect mechanism that is theoretically compatible with \textit{pareto optimality}. We provide the formula for this mechanism and then a game-theoretic treatment of it which proves its inconsistency with the impossibility results discussed earlier, and demonstrates its ability to achieve otherwise intractable outcomes. We then close with a return to economic theory and discussion of how the mechanism overcomes the more foundational theoretical problems identified by Samuelson and Hurwicz to the attainment of incentive compatibility as a general economic problem.



\section{Characteristics of a TFM \label{sec::characteristics}}

The novel characteristics of \textit{TFMs} that lead to suboptimal provision are \emph{non-excludability}, \emph{self-provision} and \emph{informational decentralization}. 

\paragraph{Non-Excludability} allows anyone to use or provision these networks on equal terms provided they are willing to pay a competitive market price. This economic characteristic underpins the technical properties of \textit{censorship resistance}, \textit{decentralization} and \textit{network resilience}: censorship requires a mechanism with the power to exclude; centralization implies barriers to entry; resilience comes from the ability to route around byzantine nodes by adding new ones to the network. Non-excludability also contributes to economic efficiency in \textit{TFMs}, as efficiency is maximized when producers build atop blocks proposed by their peers rather than orphaning them.

\paragraph{Informational Decentralization} is not the casual concept of \textit{decentralization} as used in computer science (see: non-excludability) but the economic definition offered by Hurwicz (1972) for mechanisms in which "participants have direct information only about themselves." This characteristic makes \textit{TFMs} vulnerable to the attacks, identified by Hurwicz, in which participants manipulate the informational environment that others rely on to make strategic decisions.

\paragraph{Self-Provision} allow \textit{TFMs} to support themselves without an owner, relying instead on payouts to network participants. While volunteer-run networks are theoretically possible, their designs fall outside the scope of \textit{TFMs} as transaction fees are purely redistributive. For this reason, in volunteer mechanisms the imposition of fees leads to a dead-weight efficiency loss, since any fee-level above zero is strictly suboptimal given the cost structure of the network.

These three characteristics create fundamental tensions that \textit{TFMs} struggle to reconcile. Mechanisms must permit open access without enabling sybil attacks, offer private benefits without socializing losses, and use decomposable algorithms while resisting byzantine attacks on the message-passing layer of the information environment. We can see the importance of all three characteristics from the way they form an \emph{economic trilemma} where the removal of any one property offers immediate relief to the problems created by the other two.

Understanding these characteristics allows us to identify the specific types of \textit{goal conflict} that motivate for-profit attacks on \textit{TFMs}. The first type, conflict rooted in \textit{self-interest}, occurs when participants prefer to allocate their resources differently than the mechanism designer intends, such as to purchase a different combination of goods and services across the economy as a whole. For an example of this, a user might desire to save a portion of their transaction fee to spend on a cheaper form of data transfer instead of offering a fee at the mechanism's optimal allocation. In this case, participants are signalling disagreement with the designer's intended allocation of utility, both \textit{within} the mechanism and \textit{between} the mechanism and other external goods. These attacks consequently involve participants choosing to bid at suboptimal fee levels, as they prioritize their personal preferences over the collective optimal outcome.

The second form of goal-conflict observable in \textit{TFMs} is \textit{free-riding}, which emerges when the characteristics of \textit{non-excludability} and \textit{self-provision} combine to create public goods within the consensus mechanism. While free-rider pressures are common in many mechanisms, in \textit{TFMs} they are particularly intractable due to the presence of two-sided free-rider problem where users and producers can free-ride on the mechanism in different ways: producers by maximizing the revenue they extract from any collective payout like the block reward, and users by minimizing their contribution to the security budget. As our next section explains, these are the class of attacks that manifest in the form of side-contract payments, while is why attempts to mitigate other forms of goal conflict fail to make headway disincentivizing collusion.

Our third form of goal-conflict is \textit{strategic manipulation}, which emerges because -- as Leonid Hurwicz observed -- in informationally-decentralized mechanisms participants can strategically manipulate others into suboptimally allocating their own resources by manipulating the informational space in which they form their expectations of their strategic environments. This type of goal-conflict is what incentivizes producers to create fake transactions and fake blocks, and what incentivizes users and producers to exploit threshold vulnerabilities in auction designs. This is the main problem mechanism designers have traditionally sought to eliminate through the use of techniques that attempt to achieve bayesian incentive compatibility or otherwise incentivize the truthful revelation of preferences.

As should be obvious, disagreements motivated by \textit{self-interest}, \textit{free-riding}, and \textit{strategic manipulation} are fundamentally distinct types of goal conflict. The source of the first is psychological: in the private perception of the individual that their utility is better maximized through a different resource allocation strategy. The source of the second is in the nature of the utility-providing good, whose indivisibility encourages participants to minimize their own contributions to its provision. And the source of the third is in the informational environment of the market itself, in which the costless ability for participants to mislead others can be manipulated by rational actors to induce others into forming strategies that misallocate their resources to the benefit the manipulating party.

The fact that conflicts motivated by self-interest, free-riding and strategic manipulation constitute distinct types of goal-conflict is also why each type of attack within \textit{TMFs} is expressed in different ways and through unique social dynamics. Conflicts motivated by considerations of self-interest are expressed through unilateral changes to the fees offered for transaction inclusion by the fee-paying user. Conflicts motivated by an incentive to free-ride require cooperative attempts to defund the mechanism by a subset of network participants, since at least one producer must team up with at least one user in order to enable either to free-ride on the mechanism. Conflicts motivated by strategic manipulation are adversarial and primarily involve price-manipulation strategies such as bid-shading by users or the costless inclusion of transactions that manipulate the fee-level by producers.

The fundamentally different source of the motivations to attack \textit{TFMs} and the different ways in which these attacks manifest is the primary reason that achieving fee-optimality seems like such an intractable problem. Any general solution requires the \textit{TFM} to implement an equilibrium in which none of these conflicts exist, which requires finding a technical solution that eliminates three fundamentally different motivations to byzantine behavior which can be expressed through unilateral, cooperative or adversarial strategies. It is little surprise that the existing literature has , especially in the face of auction designs that are primarily intended to addres only the third problem.

But there is a solution, which is why in our next section we pull back to economic theory to show why \textit{pareto optimality} must be the social choice rule chosen by mechanism that seeks optimal fee-throughput in a collusion-free equilibrium, as it is the only known equilibrium in which all three forms of \textit{goal conflict} do not exist.



\section{The Economics of TFM Limitations \label{sec::economics}}

Eliminating all forms of goal conflict is possible in any equilibrium that pushes production onto the \textit{utility possibilities frontier}, a curve that defines the maximum amount of utility that can be produced given the rational allocation of resources belonging to participants in the system. The relevant subfields in economics that describe this challenge deal with welfare optimality, public choice theory, informational economics and mechanism design.

In the case of welfare economics, the pioneering work on optimality was the publication of Vilfredo Pareto’s "Cours d'économie politique" (1896), which introduced the concept of pareto optimality. Pareto defined this state as one where resources are allocated so efficiently that it is impossible to improve overall social welfare by changing the way in which resources are allocated to the production of utility. Pareto optimality is thus the term used to describe any state of production that exists on the \textit{utility possibilities frontier}.

From a mathematical perspective, pareto optimality is achieved when the marginal utility derived from the last unit of each good purchased by each individual is proportional to its production cost. This implies that individuals are allocating their resources so as to maximize their utility -— every dollar is spent on whatever good or service provides the greatest marginal benefit. This allocation is considered individually rational and provides two important social critera demanded by \textit{TFMs}. First, it frees mechanisms from conflicts involving self-interest since no party will unilaterally desire to pay a greater or lesser fee. Second, \textit{pareto optimality} has attractive collusion-proof properties: if no individual can reallocate his resources without making himself worse off, no group of individuals can collude to do so without at least one member of the group suffering as a result. This eliminates all categories of user-user and producer-producer collusion.

While a \textit{pareto optimal} equilibrium eliminates all forms of goal conflict motivated by \textit{self-interest}, can such an equillibrium can be sustained in the face of \textit{free riding} pressures or \textit{strategic manipulation}?

The first question was asked by Samuelson (1954) when he observed that achieving \textit{pareto optimality} is difficult for goods with non-excludable benefits. If users can misallocate their own resources while enjoying the non-excludable forms of utility funded by contributions from their honest peers. It was Samuelson's mathematical demonstration of this problem -- that individual rationality subverted pareto optimality -- that led to the emergence of public choice theory, and prompted \citet{hurwicz1960optimality} to coin the term \textit{incentive compatibility} in reference to the opposite condition, the state in which the utility-maximizing behavior of individuals is \textit{compatible with} or leads emergently to the desired welfare condition referred to as its \textit{social choice rule}.

\LARGE
\begin{adjustwidth}{1.5em}{1.5em}
\begin{math}
\sum_{i=1}^{s} \frac{u_a^i}{u_b^i} = \frac{F_a}{F_b}
\end{math}
\end{adjustwidth}
\normalsize

Samuelson's observation -- expressed in the equation above for the \textit{utility possibilities frontier} in the generalizable two-good model -- explains why free-rider pressures subvert \textit{pareto optimality} in \textit{TFMs}. Achieving \textit{pareto optimality} requires users to allocate resources in whatever proportion keeps their costs in alignment with the utility they enjoy, yet the existence of non-excludable benefits creates rational pressures to defect. In \textit{TFMs} this manifests in the form of side-contract payments and other strategies that restrict competition for monetization of the fee. From the perspective of users, selling transactions gives producers the right to collect fees without the need to compete so intently for the privilege. Producers happily accept a lower fee as less of their own income needs flow into the collective security function as the cost of collection. This form of collusion is analogous to the classic free-rider in Samuelson's model.

On the producer side, side-contract payments permit block producers to free-ride on their peers as well. In this second case, producers offer users transaction-inclusion at suboptimal rates because private control of the transaction fee expands their share of blocks committed to the longest-chain, allowing them to extract more income from any non-excludable payout like the block reward. Once again we have a situation analogous to Samuelson's model, except in the motivation is the incentive to collect more in revenue rather than pay less in fees.

Understanding the two-sided nature of free-riding in \textit{TFMs} is critical for understanding why achieving \textit{pareto optimality} is so challenging. In the absence of this understanding, it is common to hear all forms of user-producer agreement described as suboptimal. But this is not the case! If price negotiations between users and producers drive the cost of blockspace towards \textit{pareto optimal} levels without reducing the overall funding for public good provision, they technically shift the network into a more efficient equilibrium in which fee-throughput levels are \textit{pareto optimal} and \textit{goal conflict} motivated by \textit{self-interest} is minimized. It is also trivial to see that side-contract payments can never drive transaction fees below the cost of blockspace in the absence of public goods, as rational producers cannot sustainably accept transaction fees that are lower than their cost of provision.

The inability of proof-of-work and proof-of-stake designs to contain fundamental pressures to free-ride is a major cause of suboptimality in those designs. As we shall see in our next section, these pressures are also responsible for a non-trivial number of impossibility results, since the techniques mechanism designers use to prevent \textit{strategic manipulation} -- pricing mechanisms intended to induce truthful preference revelation -- can contain those adversarial strategies but not the sort of co-operative attacks we see expressed through free-riding strategies.

The connections betwen our first two classes of \textit{goal conflict} and \textit{pareto optimality} are now clear. Conflict motivated by \textit{self-interest} exists in mechanisms that lack \textit{pareto optimality} and can be solved only by designing mechanisms that implement that social choice rule. The existence of \textit{free riding} pressures within any mechanism subverts the ability of those mechanism to sustain \textit{pareto optimality} by creating incentives for defection even when utility is being produced at optimal levels. This leaves our third category of \textit{goal conflict}, \textit{strategic manipulation}, which falls in the domain of informational economics and mechanism design. 

To put these subfields in historical context, it is useful to know that by the late 1950s and 1960s, the problems that Samuelson flagged regarding the efficient provision of public goods had become widely accepted in mainstream economics. Nonetheless, most economists still believed the production and trade of private goods under classical assumptions was more-or-less \textit{pareto optimal}. Or so they believed until 1972 when \citet{hurwicz1973design}, in his second great contribution to mechanism design, pointed out that similar problems could subvert the \textit{pareto optimal} provision of private goods in informationally decentralized mechanism.

The cause of sub-optimality that Hurwicz identified came from his study of the algorithms that other economists proposed to explain how prices move towards optimality over time. Specifically, what Hurwicz saw was that a "pre-exchange messaging step" existed in all informationally decentralized algorithms as message-passing was an essential prerequisite to the computing of expected market prices, and that the computing of this information was needed for participants to develop their welfare-maximizing strategies. As a result, in any situation where agents could costlessly manipulate price expectations they could theoretically induce others to strategically misallocate their own resources and frustrate the ability of any welfare-maximizing algorithm to approximate its intended outcome. The particular passage in Hurwicz's paper that points this out is worth quoting in full:

\begin{quote}
Economists have long been alerted to this issue by Samuelson (1954) in the context of the allocation problem for public goods. But, in fact, a similar problem arises in a "nonatomistic" world of pure exchange of exclusively private goods.... If [two parties] were both told to behave as price-takers it would pay one of them to violate this rule if he could get away with it. Now we assume that he cannot violate the rule openly, but he can "pretend" to have preferences different from his true ones. The question is whether he could think up for himself a false (but convex and monotone) preference map which would be more advantageous for him than his true one, assuming that he will follow the rules of price-taking according to the false map while the other trader plays the game honestly. It is easily shown that the answer is in the affirmative. Thus, in such a situation, the rules of perfect competition are not incentive-compatible.
\end{quote}

In this case, our form of \textit{goal conflict} does not involve participants re-allocating their own resources (\textit{self-interest}) or co-operating with others to underfund public goods (\textit{free-riding}) but involves participants adversarially and \textit{strategically manipulating} the informational environment to frustrate efficient price-discovery and induce others to misallocate their own resources. In the context of \textit{TFMs}, we see this exploited whenever producers put their own fees into blocks, whenever users engage in bid-shading, or whenever any participants costlessly loop money around the chain to create fraudulent representations of blockchain history.

Awareness of these informational manipulation is what led Hurwicz to develop his framework for studying \textit{incentive compatibility} and launch the subfield of mechanism design, which uses mathematical logic to study whether specific market structures (mechanisms) can achieve (implement) specific outcomes (social choice rules) in the presence of participants who make strategic decisions on the basis of private information. The informational nature of the problem -- caused by the assumed ability of participants to costlessly distort market perceptions -- is the reason "truthful revelation of preference" is considered such an attractive property in mechanism design, as it implies the mechanism is not vulnerable to this particularly category of goal conflict.

As an aside, since several papers on \textit{TFMs} declare \textit{incentive compatibility} impossible to achieve, it is useful to remember that Hurwicz never made this claim. As Eric Maskin later pointed out, such claims show a misunderstanding of Hurwicz' framework, since all mechanisms are by definition incentive compatible with their outcomes. What a failure of incentive compatibility means is that if private information is used as an input to form the strategies adopted by participants in any mechanism, then without an "incentive for truthfulness" those mechanisms cannot be assumed capable of implementing any social choice rule.

This carries back us to our discussion of the challenge of implementing {pareto optimality} within \textit{TFMs}. For while Hurwicz is often misinterpreted as implying that the direct revelation of preferences is a pre-condition for achieving \textit{pareto optimality}, the truth is more nuanced -- market structures still exist which lack the problems Hurwicz identified with \textit{strategic manipulation}, the key exceptions being \textit{atomistic} markets characterized by perfect competition, markets in which the utility purchased varies with price paid, and markets lacking a pre-exchange messaging step. Eric Maskin, who would later win the Nobel Prize for his work on the revelation principle, confirmed Hurwicz's intuition when he found that \textit{pareto optimality} is possible in some market structures without the need for truthful preference revelation as an intermediary step.\citet{maskin1999nash}. His revelation principle also confirms this in a more subtle way, by showing that a symmetry of outcomes must exist between mechanisms where information is computed in decomposable fashion using agent-level functions, and mechanisms where the exact same information is revealed truthfully and the computation is performed by a centralized mechanism in a non-decomposable fashion. As Maskin showed, if the centrally-computed outcome does not result in a Nash Equilibrium then the decomposable function cannot have one and at least one agent must be lying about their true preferences.

Maskin's work revealed a deeper truth: all incentive compatible mechanisms will induce the revelation of private information one way or the other, meaning that the difference between mechanisms is not whether they reveal user preferences so much as whether they reveal those preferences \textit{directly} or \textit{indirectly}. In direct mechanisms participants share private information truthfully in the pre-exchange negotiation step, allowing composable algorithms like auction mechanisms to calculate optimal allocation strategies and carry them out. In \textit{indirect} mechanisms participants reveal their preferences either obliquely in the price-discovery process (such as by preference-ranking bundles of goods) or by skipping the price-discovery stage and submitting purchase orders directly onto the market.

As a result, given that \textit{TFMs} are \textit{informationally decentralized} mechanisms in which users and producers make strategic decisions on the basis of private preferences over resource allocation, if our social choice rule is \textit{pareto optimality}, we cannot achieve it in any mechanism where participants can costlessly manipulate any information needed to estimate market pricing for transaction inclusion. If a mechanism permits block producers to costlessly include their own transactions in blocks we thus have \textit{de facto} grounds for concluding that incentive compatibility with \textit{pareto optimal} fee-throughput is impossible to sustain. Conflict motivated by \textit{strategic manipulation} can only be eliminated by making the inclusion of self-generated transactions costly, such that the decision by a block producer or user to use the blockchain reveals private information that the mechanism can exploit to shift overall provision into a more efficient equilibrium.

Hurwicz (1973) provides several other conditions any \textit{TFM} will need to meet in order to successfully implement \textit{pareto optimality}. The first is that one-shot mechanisms are insufficient, since algorithms with \textit{inertia} are required to iterate price levels into their optimal positions over time. This suggests that the information required to calculate the price of blockspace must be observable from the environment rather than collected from peers. And as Jordan (1986) observes, some form of smoothing of costs or payouts is beneficial to prevent mechanisms unpredictably oscillating around the desired equilibrium point.

In summary, our three types of goal conflict -- self-interest, free-riding, and strategic manipulation -- are distinct issues that affect most \textit{TFMs}. The first can only be eliminated building a mechanism that implements \textit{pareto optimality} as its social choice rule. But the second and third types are known in economics to prevent mechanisms from sustaining that equilibrium. All three forms of goal conflict thus induce existential attacks on the sustainability of the consensus mechanism. A threshold user who underbids in a Vickrey-Clarke-Groves auction is exhibiting self-interest. Users who conspire with producers to defund the security budget are free-riding on their non-colluding counterparts. A block producer who floods the network with spam transactions to drive up fees is engaging in strategic manipulation. 

A full solution requires eliminating all three kinds of goal conflict: killing free-riding pressures are as essential to security as imposing a cost on the publication of false market data.

With this economic framework in place, in the following section we turn our attention to the existing literature on \textit{TFMs} in computer science, with the goal of showing why the impossibility results in these papers reflect the limitations of their models and approaches rather than the limits of what is possible in distributed systems.



\section{The TFM Literature in Computer Science \label{sec::distsys}}

To our knowledge, this is the first paper to show how goal conflict prevents \textit{TFMs} from achieving \textit{pareto optimality} and makes auction models informationally incapable of resolving conflict within \textit{TFM}. In order to understand the exact problem, this section reviews how computer scientists have studied this issue to show the general issues with approaches used.

Early attempts to model \textit{TFMs} as auctions were Bitcoin-specific, starting with "Redesigning Bitcoin's Fee Market", which proposed using a "monopolistic auction" to stabilize miner revenue, followed by Andrew Yao's "An Incentive Analysis" which showed this maximized miner revenue at scale. Basu, Easley, O’Hara, Sirer then proposed a modified Vickrey-Clarke-Groves mechanism as a better choice for maximizing the collective welfare of both users and miners.

While all three papers focused explicitly on Bitcoin, the concerns over efficiency showed awareness \textit{TFMs} are not just resource allocation mechanisms, but are themselves subject to conflict over resource allocation within the broader economy! Computer science was on the cusp of seeing the underlying economic nature of their problem, identified by Hurwicz in 1973 as "goal conflict", and realizing that \textit{pareto optimality} would be the social choice rule required to solve it.

Computer science pulled back slightly in 2021 when Tim Roughgarden\citet{roughgarden2021,roughgarden2024transaction} offered a paper that modelled Transaction Fee Mechanisms (TFMs) as two-sided auctions in which block producers are given a temporary monopoly over the production of a block and must strategically allocate a subset of transactions into it. Looking beyond Bitcoin towards a landscape of competing consensus mechanisms, Roughgarden returned to characterizing the incentive-alignment issue as resulting from internal rather than economy-wide conflict over resource allocation. He was the first to highlight the difficulty of achieving incentive compatibility for both users (UIC) and block producers or miners (MIC), leading to seminal works~\cite {roughgarden2021,roughgarden2024transaction} on the limitations of Bitcoin's "first-price auction" and Ethereum's EIP-1559~\cite{buterin2019eip} among others. Roughgarden~\cite{roughgarden2021,roughgarden2024transaction}.

Since 2021, the vast majority of academics working on \textit{TFM design} have followed Roughgarden in modelling \textit{TFMs} as two-party auctions in which producers clash with users over how to allocate blockspace. The attractiveness of the approach is obvious: it focuses on internal rather than external motivations for conflict, it targets an essential step in the formation of consensus, and it uses a two-sided game that is tractable to model. Significantly, Myerson's lemma and virtual valuations can also be used to generalize the rational strategies of participants in these games so they can be asserted to hold in larger games with many players. Unfortunately, the work is simply producing a series of impossibility results.

Since one of the purposes of this paper is to present a mechanism that evades these problems, it is useful show how this choice of modeling \textit{fee mechanisms} has created structural incompatible with a productive solution. In this light, the first problem is methodological treatment of UIC and MIC as properties which can exist outside the context of a social choice rule. Instead of identifying an equilibrium like pareto optimality that guarantee both fee-optimality and collusion-resistance for users and producers alike, and asking what private information both participants would need to disclose for any \textit{direct mechanism} like an auction to achieve it, the literature assumes that truthful preference revelation is a sufficient goal in-and-of-itself. This is typically done by citing the Revelation Principle and observing that any mechanism capable of achieving a nash equilibrium must have an equivalent in which (see Roughgarden p. 13) truthful bidding is a dominant strategy.

The problem with this assumption is that the preference information any algorithm needs to be revealed depends on the social choice rule at stake, and specifically on whether we are in the presence of a problem that requires high-dimensional preferences to calculate.

Viewed sympathetically, we can intuit that the field's implicit social choice rule is an "efficient allocation" of blockspace. This seems fair to assume given Roughgarden's citation of the Vickrey-Clarke-Groves (VCG) mechanism as being UIC and the lack of any seeming challenge to this assumption. If this auction is considered to reveal truthful information sufficient for optimizing participant utility in one mechanism, it seems intuitive that it would collect the same information needed to optimize utility in a different mechanism, but the information required actually depends on the social choice rule and the difference between the types of conflict mechanisms are intended to address requires a very different type of "utility" information to be collected in our case.

Note, for instance, that the VCG auction is a \textit{direct mechanism} that does not require high-dimensional preference information as part of its process of truthful preference revelation. Users share information on the maximum price-point at which they are willing to purchase the single privte good being allocated given a fixed price and production schedule for everything else, not their comparative preference for how to divide their resources between all goods and services competing for consumption of the same transaction fee at all viable price equilibria as required for implementing \textit{pareto optimality}. The VCG auction is thus informationally inadequate for eliminating byzantine strategies motivated by "self-interest" -- our first class of incentive to suboptimality. Similarly, the VCG auction has no informational basis for combatting \textit{free-riding}, since those are cooperative strategies to defund the production of a form of utility not covered by models that treat blockspace like a private good.







Roughgarden's papers were quickly followed by papers from Elaine Shi and Hao Chung, who offered technical definitions like "side-contract proof" ("no utility increase from off-chain payments") rather than using Roughgarden's technical definition of OCA-Proof. The difference between the two is essentially the difference between whether collusion maximizes revenue in the context of a single block or across potential forks in a chain. The concept of OCA-Proof thus encompasses types of collusion that lead to block orphaning while SCP-based approaches do not. Pareto optimality eliminates both possibilities on the fundamental grounds that there is no rational strategy for colluding in either case. The mechanism proposed later in this paper also elegantly sidesteps Roughgarden's concerns that any "fee burn" must invite collusion because "because OCAs allow the miner and users to coordinate and evade the intended burn." This is of course not possible in routing work mechanisms where the burn is the cost of producing a block, as it cannot be evaded by moving the payment off-chain.





A more subtle problem applies to the treatment of block producers, who are simply asked to implement the fee mechanism. The lack of any need for producers to reveal private information raises questions about why we are modelling this game as a two-sided strategic interaction, and points to a deeper methodological problem. For as noted in our first section, the class of \textit{TFMs} we are studying contain dual-sided free-rider problems. This specific class of vulnerability makes it impossible to achieve pareto optimality for both parties if we require truthful preference revelation from only one party. For both parties have private incentives to adopt byzantine strategies that are driven by a desire to free-ride on their peers. Eliminating collusion thus requires either eliminating collective action problems generally (and the auction mechanism cannot handle this as it focuses exclusively on a single private good) or by identifying a kind of "private information" which can by leveraged by a mechanism to motivate producers to shift their strategies away from defunding fection and towards cooperation. By denying producers the ability to act strategically on the basis of private preferences, modelling blockchains as auction mechanisms leads inescapably to impossibility results as they prevent the most critical party from behaving strategically!

The third and most fundamental problem with the auction model generally is that it is impossible to generalize its impossibility results, since the existence of an impossibility proof for this specific type of \textit{direct mechanism} can never eliminate the possibility that an \textit{indirect mechanisms} might exist that can achieve the desired results through the solicitation of a different kind of preference. Since this is a somewhat subtle point, note that while Maskin's relevation principle teaches us that all nash equilibria which are reachable by \textit{indirect mechanisms} can be implemented as \textit{direct mechanisms}, the opposite is not true. So even if the auction model was informationally appropriate for implementing \textit{pareto optimality}, we cannot conclude from an impossibility proof generated assuming the limitations of a direct mechanism that no indirect mechanism exists which is capable of skirting that problem.

Understanding this point is important for seeing how the mechanism described later in this paper solves the problem. For Maskin's revelation principle is based on logical reasoning about the consistency of outcomes between decomposable algorithms (where participants compute their preferences privately) and composable algorithms (where users reveal their preferences to a centralized mechanism that does the work for them). In situations where the amount of information required to calculate an optimal solution is so large as to make disclosure impractical or impossible to calculate in a centralized mechanism, such as exists with the high-dimensional preference data needed to compute \textit{pareto optimal} equilibria in informationally decentralized environments, \textit{indirect mechanisms} that use \textit{decomposable algorithms} to filter and transform participant preferences prior to their revelation can be informationally necessary to achieve incentive compatibility. It should be noted that Maskin's revelation principle still holds -- truthful preference revelation happens in both types of mechanism -- but it can happen in a different stage, either in the "action stage" identified by Hurwicz where bids are submitted directly to the market, or obliquely in the "pre-exchange negotiation stage" in a more indirect and filtered form.

The presence of public goods in consensus mechanisms is what forces the need for high-dimensional preference measurement, as they pull focus away from maximizing the kind of "single well-defined objective function" Hurwicz associated with computational models and towards the more complicated multi-variate forms of goal conflict suitable for economic analysis. Perhaps because of this, it is not surprisingly to see mechanism designers with stronger economic backgrounds explicitly recognize the presence of public goods, as is the case in a recent paper by Elijah Fox, Mallesh Pai, and Max Resnick on "Censorship Resistance in On-Chain Auctions". While the assumptions these authors make are not strictly true -- transactions fees induce both private and public goods and only incentivize the provision of public goods to the extent they circulate openly for competitive inclusion -- these authors are absolutely correct that off-chain payments involve a form of free-riding and addressing this problem is the key challenge for mechanism designers.

While the proposal by Fox fails on technical grounds (the degree of competition for fee collection can be manipulated by collusion in any non-excludable mechanism), their insight helps explain why \textit{indirect mechanisms} are traditionally used in economics when optimizing resource allocation to non-excludable goods and eliminating free-rider pressures. \textit{Indirect mechanisms} are the preferred approach for solving these class of problems, such as in the curious case of the Clarke-Groves mechanism (not to be confused with the VCG mechanism), a indirect mechanism in which users are asked to submit bids across bundles of goods, some of which may include public goods. Given the parallels between the information requirements to solve both problems, it is likely no accident that the solution this paper identifies is an \textit{indirect mechanism} that leverages decomposabitility to avoid the need for truthful preference revelation during the "pre-exchange negotiation step" as a necessary precondition for achieving incentive compatibility.

Returning to our review of the related literature, a second influential string of papers has come from Hao Chung and Elaine Shi in their work on \textit{side-contract payments} and the \textit{zero-revenue bound}. Specifically, Hao and Chung advance claims that side-contract payments (SCP) are impossible to disincentivize in any mechanism where the income for block producers is above zero.

The same methodological problems apparent elsewhere replicate here, as Hao and Chung treat truthful preference revelation as if it is a valid social choice rule rather than an intermediary step to achieve one in the presence of private information. The most significant difference with this approach is that unlike academics who view an "efficient and fee-stable blockspace allocation" as the implicit social choice rule, for Hao and Chung it is the possibility for a collusion-free environment that takes center stage, with results suggesting that collusion is impossible to eradicate in \textit{TFMs} with revenue above the \textit{zero-revenue bound}.

The framework provided above provides an intuitive explanation of why Hao and Chung stumble into their zero-revenue bound. As explained in Section 2, the underlying source of suboptimal forms of user-producer collusion is the existence of dual-sided free-rider problems embedded in the mechanisms. Their results follow deductively from this problem. At any positive fee-level users have an incentive to collude with producers to free-ride on the contributions of their peers to the security budget. This problem can be avoided by compensating producers through an inflationary block reward, but that invites producers to free-ride on the supply-side payout. Avoiding one trap pushes us into the other, so the only situation in which we avoid collusion completely in their model is if neither fees nor block rewards exist.



What percentage of the remaining papers are writing about blockchains and what percentage are simply writing about auctions? Making similar assumptions as their predecessors (auction model, no clear social choice rule, costless manipulation of informational environment), Aadityan Ganesh, Clayton Thomas and Matthew Wienberg not surprisingly end up in the same place, with the value of their work consisting mostly of several new terms like "off-chain influence proofness" that capture specific forms of collusion. While the authors identify "external opportunities" for profit not captured within the fee mechanism (implying goal conflict), they fail to follow their observations to their obvious conclusions: that the auction model itself is an inappropriate tool for analysing this problem. But the proof-of-stake models they study cannot address any of these problems. So how could they -- or any of their peers -- be expected to find a solution, when their focus is examining mechanisms designed by developers who also fail to understand the underlying problems they face?

There are some positive results, interesting primarily for showing that market mechanisms -- not auctions -- hold the key to solving these problems. Rejecting the tendency to treat auctions as one-shot games,CITATION find that repeated-games. This works for the same reason that \citet{ferreira2021dynamic} finds -- it creates the form of "inertia" required in free markets for price levels to move closer to pareto optimal levels in equilibrium. Ferreria's finding also shows the benefit of moving pricing information into the state of consensus itself, minimizing opportunities for \textit{strategic manipulation} by shifting the market price into the environment rather than making it only accessible through unreliable peer messages.

The tendency in computer science papers to treat the conclusions of earlier papers as axioms in new lemmas intended to develop new theorems has exacerbated the tendency for impossibility results to be exaggerated and amplified. 

%%%
%%% \citet{ferreira2021dynamic} propose a posted-price TFM and provide an equilibrium posted-price at which TFMs are incentive compatible for both users and producers. \citet{chen2022bayesian} propose a TFM that guarantees BUIC and collusion resistance.
%%%
\citet{chung2023foundations} introduce a "Burning Second-price" TFM that compromises allocative efficiency to guarantee user and block producer IC. In their model, the authors tweak the utility model with "$\gamma$-Strict" utility for users/producers. The new model captures the future cost of introducing fake transactions discounted by a \textit{public} parameter $\gamma\in [0,1]$. We believe compared to  "$\gamma$-Strict" utility \ourTFM's incentive rule introduces a natural cost for introducing fake transactions to the users/producers. Moreover $\gamma$-Strict utility does not prevent free-riding.
%%% \citet{gafni2024barriers} fully characterize deterministic TFMs and show that only the trivial TFM -- that never confirms any transaction -- is IC for users and producers and is simultaneously collusion proof. \citet{chung2024collusion} shows the impossibility of satisfying Block Producer IC, User IC, and user-producer collusion-proofness in randomized TFMs. They also propose relaxations to circumvent the impossibility. Among other works, \citet{wu2023maximizing} provide an optimal bound for the block producer for a TFM that is Bayesian IC and collusion-proof, assuming that a known fraction of the users are honest. \citet{damle2024designing} instead focus on reducing transaction fees by redistributing the surplus while providing a relaxed incentive-compatibility guarantee.
%%%
%%% % \varul{
%%%\paragraph{Maximal Extractable Value (MEV).}
%%%Another developing line of work in TFMs decouples the process of block production and extracting maximal value out of the transactions (referred to as "maximal extractable value" (MEV)). \citet{bahrani2023transaction} discuss the incentive and welfare properties for Block Producers and Users considering the additional MEV value as "block producer surplus". \citet{bahrani2024transaction} introduce another type of agent, namely \emph{searchers}, that act like MEV Oracles and analyze TFMs with MEV. \citet{bahrani2024centralization} discuss the problem of centralization due to heterogeneity in knowledge among different agents. 
%%% }
%%%Informationally, what these papers seem to be establishing is that in 
%%%






\section{A Solution \label{sec::solution}}

This section introduces \ourTFM: a \underline{R}outing \underline{T}hreshold-based \underline{R}andomized-TFM.

\ourTFM is a Dutch clock auction where producers compete for the right to produce blocks through the collection of transactions and the submission of their fees into a burning mechanism. A costly lottery which follows the production of each block has the potential to resurrect and redistribute these burned fees, with this same lottery providing wrap-around sybil-resistance for the chain. The economic innovation of the approach is that it makes the production of blocks costly for participants who spend their own fees.

\ourTFM incentives prices to move towards \textit{pareto optimality} by punishing the two forms of activity that push any \textit{TFM} away from optimal fee-levels: the deliberate censorship of transactions which pay theoretically competitive fees, and the inclusion of \textit{fake transactions} whose inclusion is motivated by a desire to manipulate fee-levels.

In the section that follows, we provide game-theoretic characterization of this mechanism. This is accomplished by modelling what Hurwicz referred to as the \textit{formula} or mathematical properties of the approach. We follow this characterization with several game-theoretic proofs that this mechanism evades the impossibility results cited previously, as an \textit{indirect mechanism} that implements \textit{pareto optimality} as its social choice rule.

\subsection{Game Theoretic Characterization}

In \ourTFM, when users send transactions to nodes in the network, they include cryptographic routing signatures indicating the first \emph{hop} node. Each node adds its signature as it \emph{propagates} the transaction deeper into the network, creating within each transaction an unforgeable record of the path the transaction has taken from the user to any block producer competing to offer inclusion.

The "routing work" used to purchase blocks is derived from this chain of signatures. Specifically, the amount of routing work that is available to a producer from any transaction is given by $c\cdot \frac{1}{2^{h-1}}$, where $c$ is a network-determined constant and $h$ is the node's hop for that transaction. E.g., a node hearing about a transaction at its third hop receives $\frac{c}{4}$ routing work for that transaction. Each node gathers transactions until they have enough total routing work to meet a network-determined \textit{difficulty} \textit{threshold}, $\tau$. At this time, the node may become a block producer and broadcast a block with its set of transactions whose total routing work crosses $\tau$.

The existence of multiple nodes processing transactions allows us to model \ourTFM\ as a game with a set of $m\in\mathbb{N}$ producers, $\mathcal{P}:=[m]$. We consider each producer $i\in \mathcal{P}$ to be \emph{myopic} and \emph{strategic}. To simplify analysis, we assume that each transaction is of the same size, with each block's capacity denoted by $k \in \mathbb{N}$. Furthermore, we let $n\in\mathbb{N}$ denote the total number of users, denoted by $\mathcal{U}:=[n]$. We assume that each user $j\in\mathcal{U}$ is also myopic and strategic~\cite{roughgarden2021,ferreira2021dynamic,chung2023foundations,gafni2024barriers,damle2024designing}. A user $j\in\mathcal{U}$ is interested in getting a slot in the block for its transaction. Let $\theta_{j}\in \mathbb{R}_{\geq 0}$ denote user $j$'s private valuation for its transaction's confirmation and $b_j \in \mathbb{R}_{\geq 0}$ as its transaction's public bid.

As is common in distributed consensus mechanisms, each block producer $i\in \mathcal{P}$ has its private copy of the set of outstanding transactions, known as \emph{mempool}. The presence of routing signatures within transactions means that in \ourTFM\ producers store both the transaction bids and the specific hop at which they received the transaction. That is, producer $i$'s mempool is the tuple $\mathcal{M}_i=({F}_i, {H}_i)$. $\mathcal{M}_i$ comprises the set of user bids $F_i = (b_1,\ldots, b_{n})$ and their corresponding hops $H_i = (h_{i,1},\ldots, h_{i,n})$.

This lets us define the routing work for any transaction $(b,h) \in \mathcal{M}_i$. Consider a function $\omega:\mathbb{R}_{\geq 0}\times\mathbb{N}\rightarrow\mathbb{R}_{\geq 0}$ that represents the amount of routing work gained by a block producer at the $h\textsuperscript{th}$ hop. In \ourTFM, the routing function $\omega$ is:

\begin{equation}
    \omega(h) := c\cdot2^{1 - h}
\end{equation}

That is, \ourTFM\ offers 1\textsuperscript{st}-hop nodes $c\in \mathbb{R}_{\geq 0}$ units of routing work, 2\textsuperscript{nd}-hop nodes $\frac{c}{2}$ units of routing work, 3\textsuperscript{rd}-hop nodes $\frac{c}{4}$ units of routing work, and so on.

The algorithm for calculating the routing work available to block producers allows us to provide the \textbf{optimization function}, denoted by \ourOPT, which involves each production $i\in \mathcal{P}$ selecting transactions from $M_i$ for inclusion in their proposed blocks:

\begin{align}\tag{\ourOPT}
    \begin{rcases}
        \underset{S\subseteq \mathcal{M}_i}{\mbox{arg}\max} & \min_{(b_t,h_t)\in S}  b_t \\
       \mbox{{s.t.}}\  & \sum_{(b_t,h_t)\in S} \omega(h_t) \geq \tau \\
       & |S| \leq k\\
   \end{rcases}
\end{align}

The first constraint ensures that $S\subseteq \mathcal{M}_i$ clears the \textit{network-determined threshold for routing work}, $\tau$\footnote{The threshold $\tau$ is a network-determined dynamic parameter and increases upon block production, and slowly decreases until the next block is produced as similar to other Dutch clock auctions, similar in principle to the role of the base fee in EIP-1559~\cite{buterin2019eip}. As we consider myopic block producers and users, we omit additional details on the role of $\tau$.}. For the second constraint, recall that each transaction is of the same size. This implies that the total transactions in a block cannot exceed its capacity, $|S| \leq k$. Throughout this paper, we refer to $S\subseteq \mathcal{M}_i$ as the subset that satisfies these two constraints and $S^\star$ as the solution to \ourOPT.

As follows, a producer $i \in \mathcal{P}$ computes $S \subseteq \mathcal{M}_i$, such that the transactions in $S$ clear $\tau$, or,
$$
\sum_{(b_t,h_t) \in S} \frac{c}{2^{j-1}} \geq \tau.
$$

In order to keep block production stable over time, the consensus mechanism adjusts $\tau$ over time to target a constant pace of block production. If fee-throughput increases, $\tau$ is increased to force blocktime back into the desired pace by making block production more expensive. If fee-throughput decreases, $\tau$ is reduced slightly to make block production cheaper.

We now progress how to payouts are issued. As distinct from other networks, the first thing that happens after a block is produced is that all of its fees are removed from circulation, and then a lottery takes place to distributing up to half of those fees back into circulation as a payout to a random network node. The burning of fees can be done in a pure implementation by having the consensus mechanism simply destroy half of the tokens collected in network fees. A more practical implementation can use a costly method of random-number generation such as hashing to power a post-block payout lottery and give miners half the block reward. Penalizing fee-throughput spikes is also helpful. Given that this paper focuses on the forumula for routing work, we set the fraction of network fees that are burned in \ourTFM  as $1/2$, i.e.

\begin{equation}\label{eqn::burn}
    \delta(S) := \frac{1}{2} \sum_{(b_t,h_t)\in S} p_t
\end{equation}

As an aside, while it is not necessary for \ourTFM to have a second-price payment rule, we adopt it here for the convenience of demonstrating UIC, since one of the purposes of this section is to show that previous impossibility results that make similar assumptions no longer apply to routing work mechanisms. Under this second-price payment rule, the payment collected from \emph{each} user whose transactions are confirmed in $S$ is the lowest winning bid (say) $p$. The total payment collected is $\frac{1}{2}\cdot|S|\cdot p$ (recall that the other half is burned).

Whether a second-price payment rule is used or not, the lottery that determines the winner of the payout begins after the production of the block. This lottery first selects a random transaction from within the block, and then a random node from within the routing paths of the selected transaction. 
%%% This lottery is defined formally in Figure~\ref{fig::payment}.

\begin{figure}[h]
\centering
    \fbox{
     \centering \parbox{0.9\linewidth}{  % Box width
     The revenue, $\frac{1}{2}\cdot|S|\cdot p$, collected when a block is produced is given to the winner sampled from the following distribution. All sampling is done on-chain, i.e., in a trusted manner~\cite{chung2023foundations,damle2024designing}.
        \begin{enumerate}[leftmargin=*,itemsep=0.2em]
            \item Sample a transaction $t^\star \in S$ uniformly, i.e., $t^\star \sim \text{Uniform}(S)$.
            \item From the routing path of $t^\star$, sample a node through a probability distribution that weighs each node by their share of the routing work available at their hop over the total sum of routing work available to all nodes in the routing path of the transaction as included in the block.
                \begin{itemize}
                    \item Let the producers part of $t^\star$'s routing path be (w.l.o.g.) $\mathbf{P}_{t^\star}=\{1,\ldots,l\}$.
                    \item Any producer's $i \in \mathcal{P}_{t^\star}$ routing work for the transaction $t^\star$ is $\omega(t^\star;h_{i})$. Likewise, the total routing work for $t^\star$ is $\sum_{i \in \mathcal{P}_{t^\star}} \omega(t_\star;h_{i})$.
                    \item We sample a producer $i^\star \in \mathcal{P}_{t^\star}$ from the following weighted probability distribution:
                        $$
                        \Pr(i^\star) \sim \frac{\omega(t^\star;h_{i^\star})}{\sum_{i \in \mathcal{P}_{t^\star}} \omega(t_\star;h_{i})}
                        $$
                    \item The producer $i^\star$ receives the payment $\frac{1}{2}\cdot|S|\cdot p$.
                \end{itemize}
        \end{enumerate}
      }
    }
    \caption{\ourTFM: Revenue Lottery given $S$ (refer \ourOPT)}\label{fig::payment}
\end{figure}

For an intuitive example, if a transaction is sampled that has two nodes in its routing path, the total routing work for all nodes in the routing path is $c+\frac{c}{2}=\frac{3c}{2}$. The sampling probability of the first-hop node is $\frac{c}{3c/2}=\frac{2}{3}$ while the sampling probability of the second-hop node is $\frac{c/2}{3c/2}=\frac{1}{3}$. 

This allows us to define the probabiltiy of an arbitrary producer $i$ winning the lottery, which depends on the efficiency with which it sends fees into the burning mechanism, denoted by $\alpha_{i}, as:

\begin{equation}\label{eqn:alphai}
    \alpha_{i} = \sum_{j = 1}^{m} \Pr(\mathbb{I}_{i} = 1 | S_{j})\cdot \Pr(S_{j})
\end{equation}

Here, the indicator variable  $\mathbb{I}_{i} = 1$ denotes the event that producer $i$ is selected as the winner (recipient) of the block's payment; $\mathbb{I}_{i} = 0$ otherwise. $\Pr(S_j)$ denotes the confirmation of the set $S_j$ owned by the $j^{\mbox{th}}$ producer.

The dynamics of the routing work mechanism provide security against classes attacks of attacks which are impossible to eliminate in other mechanisms. Producers minimize their losses in the payout lottery if they spend their own fees, but doing so also burns half of their own money. Adding transactions which have been routed by other nodes adds fees that can subsidize the unlock cost, but also introduce competing claims-on-payout from other routers that grow faster than the work provided. As our next sections will show, in a competitive dynamic this lose-lose situation dissuades rational producers from using their own money to extend the chain once the network is in equilibrium, ceteris paribus.

\subsection{Incentive Compatibility}

The standard way TFM papers test for UIC and MIC is to establish incentive compatibility for users following Myerson's Lemma, and then examine whether producers have an incentive to faithfully implement the mechanism assuming that the probability of block production -- and thus the utility offered to users for transaction inclusion -- is held constant. In this section we take the same approach to prove the impossibility results of earlier papers do not apply to \ourTFM.

\paragraph{User Incentive Compatibility}. As mentioned above, Myerson's Lemma \cite{myerson81} provides a condition under which any mechanism (like an auction) ensures users bid the maximum amount they are willing to pay irrespective of what every other user does. According to the lemma, the allocation rule must be monotone in the user bids, given other bids are constant. Further, it must follow the proposed payment characterization. E.g., it is well known that the generalized second-price auction (or VCG) is a special case of Myseron's Lemma and thus incentive compatible for users. The TFM literature considers the single-demand, homogeneous setting, i.e., each user has a requirement of at most one item, and all the available items are copies of a single item. The VCG auction allocates to the highest $k$ users and charges them the $(k+1)^{th}$ bid. 
 
In \ourTFM, the block producers must consider both the bids and the routing work corresponding to each transaction. Due to the additional requirement of the routing work threshold, producers may not follow the standard VCG allocation. That is, the highest $k$ bids may not clear the routing work threshold if they have propagated deeply into the network and their transactions provide less "routing work" for the production of a block. Therefore, in order to demonstrate that Myerson's Lemma holds we must first show that the proposed allocation rule is monotonic.

\begin{lemma}\label{lemma:monotone} 
    For any user $i\in \mathcal{U}$, \ourTFM\ allocation rule $\bm{x}$ is monotone with respect to their bid (transaction fees), given the remaining bids $\mathcal{U}\setminus \{i\}$ do not change. 
\end{lemma}
\begin{proof}

A strategic producer selects transactions that clear the routing work threshold and satisfy the block capacity constraint, captured by \ourOPT's feasibility constraints. Note that, the routing work of any transaction is independent of the user's bids. Let $\mathcal{S}$ be the set of the subset of feasible transactions. The producer selects the subset that maximizes the minimum bid (objective of \ourOPT). If a user's transaction belongs to any feasible subset, increasing the bid will have the following effect. 

If the said bid is the minimum in $S$, increasing it will increase the chance of confirmation. It will not affect the chance of confirmation if it is not the minimum in $S$. Changing the bid does not have any effect if the transaction does not belong to any feasible subset (due to the constraints in \ourOPT). Hence, the allocation is non-decreasing with increasing bid.
\end{proof}

We note that \ourTFM\ has a monotonic allocation rule, it does not entirely satisfy Myerson Lemma's~\cite{myerson81} payment characterization in the absence of a price-setting transaction. As we have yet to establish that it is costly for producers to include their own price-setting transactions in the block. Therefore, similar to \citet{roughgarden2021}, we suggest using the minimum bid in $S^\star$ as the price-setting bid. Theorem~\ref{thm:UIC} shows that this payment rule ensures almost URC. That is, when there are sufficient transactions and the difference between transaction pairs is small, the incentive from deviating is negligible.

\begin{theorem}\label{thm:UIC}
   \ourTFM\ is incentive compatible for users
\end{theorem}
\begin{proof}

    We prove UIC through a case-by-case analysis.

    Let $S^\star$ be the block producer's optimal subset of transactions based on the bids, computed via \ourOPT. The utility to the user is the value of inclusion in the blockchain at the level of security generated by the user if they bid their true value.

    Let $B = \min_{(f, h)\in S^\star} f$ be the minimum accepted transaction.

    \begin{itemize}[leftmargin=*]

        \item \textbf{Case 1.} $\theta_i < B$ for any user $i$, if $b_i = \theta_i$ the user does not get selected in $S^\star$ and gets zero utility. If the user under-bids, i.e., $b_i < \theta_i$ the utility remains zero. Upon overbidding, i.e., $b_i > \theta_i$, the user might get selected, but the user's utility will be $\theta_i - B < 0$. For Case 1, bidding true value maximizes the utility.

        \item \textbf{Case 2.} $\theta_i > B$, if $b_i = \theta_i$ and $b_i \in S^\star$, i.e., the user is truthful and other constraints (independent of bid) ensures the selection of $i$ and utility of $\theta_i - B$. As long as the bid value $b_i > B$, the user might get a utility $\theta_i - B$. If the bid $b_i < B$, the utility will be zero. Hence, the maximum utility is obtained at truthful bidding. In the other scenario where $b_i = \theta_i$ and $b_i \not\in S^\star$, i.e., the user does not get included due to other constraints, the user's utility is zero. Changing the bid does not impact its inclusion; thus, the utility remains zero.

        \item \textbf{Case 3.} $\theta_i = B$, in this case, the user can deviate by bidding the lowest value needed to qualify for $S^\star$. Since this deviation explicitly lowers fee-throughput relative to the optimal level at which user utility is assumed, $\tau$ is lowered by consensus and the amount of utility received by the user is also lowered. As per our starting assumptions, this is a suboptimal outcome as the reduction of the fee is not costless in terms of the utility purchased and the user is in a suboptimal equilibrium -- if they preferred this equilibrium they should have bid it originally as per the Revelation Principle.
    \end{itemize}
    This proves the theorem.
\end{proof}

\paragraph{Producer Incentive Compatibility}. The standard way in which MIC is examined is to demonstrate that block producers with a temporary monopoly over block production have incentives to manipulate fee-levels. In this section we show the same assumptions other papers treat as universal limitations lead to different results in \ourTFM. To do this, given the presence of a routing payout and the potential for strategic attacks on it, we first show that RTR-TFM incentivizes producers to propagate transactions without engaging in malicious routing strategies: either the hoarding of transactions or the addition of fake identities on the routing network. We then show that the inclusion of fake transactions is only rational if it pushes the network towards a \textit{pareto optimal} equilibrium, and thus constitutes a form of strategic behavior that the mechanism leverages to achieve fee-optimality.

\begin{lemma}\label{lemma:routing}
    In \ourTFM, routing is a Dominant Strategy over hoarding transactions for any block producer $i\in\mathcal{P}$. 
\end{lemma} 
\begin{proof}

    Consider four block producers, say $A_{1}, A_{2}, B_{1}, B_{2}$, such that $A_{1}$ and $A_{2}$ are connected (i.e., messages from $A_{1}$ reach $A_{2}$ in single hop). Also, consider $B_{1}$ and $B_{2}$ as connected. We assume $A_1$ and $B_1$ receive the same transaction as first hop nodes. Now, we examine 2 cases: (1) when $B_{1}$ hoards transactions, and (2) when $B_{1}$ routes transactions. We show that, in either case, $A_{1}$ receives a higher utility on routing than hoarding.
    
    For the proof, we quantify $u(A_1\ \mbox{routes} | B_1\ \mbox{hoards})$ as the utility $A_1$ receives from routing the transaction in the event $B_1$ decides to hoard it. Further, $u(A_1\ \mbox{hoards} | B_1\ \mbox{hoards})$ denotes the utility for $A_1$ when both choose to hoard. Likewise, $u(A_1\ \mbox{hoards} | B_1\ \mbox{routes})$ and $u(A_1\ \mbox{routes} | B_1\ \mbox{routes})$ correspond to utilities for $A_1$  when $B_1$ decides to route to $B_2$.
    % \begin{itemize}[leftmargin=*]
    \smallskip

    \noindent \textbf{Case 1: $B_1$ hoards the transaction.} If $A_1$ hoards then the probability of $A_1$ and $A_2$ producing the block is $\Pr(A_1) = \Pr(A_2) = \frac{1}{2}$, that is, both are equally likely. Let $p$ be the payment received, implying $A_1$'s utility is $u(A_1\ \mbox{hoards}) = \frac{1}{2}\cdot p$. When $A_1$ propagates instead of hoarding and given $\Pr(A_1) = \Pr(A_2) = \Pr(B_1) = \frac{1}{3}$, i.e., all the three nodes involved are equally likely to produce a block, $u(A_1 \ \mbox{routes}) = \Pr(A_1) \cdot p + \Pr(A_2) \cdot \frac{2}{3} \cdot p = \frac{5}{9} \cdot p $. Thus $u(A_1\ \mbox{routes} | B_1\ \mbox{hoards}) > u(A_1\ \mbox{hoards} | B_1\ \mbox{hoards})$.

   \smallskip
    \noindent \textbf{Case 2: $B_1$ routes the transaction to $B_2$.} If $A_1$ hoards then $u(A_1\ \mbox{hoards}) = \frac{1}{3} \cdot p$ where $\Pr(A_1) = \frac{1}{3}$. If $A_1$ decides to route to $A_2$, and given that all the four nodes involved are equally likely to produce the block, we get  $u(A_1\ \mbox{routes}) = \Pr(A_1) \cdot p + \Pr(A_2) \cdot \frac{2}{3} \cdot p  = \frac{1}{4} \cdot p + \frac{1}{4} \cdot \frac{2}{3} \cdot p = \frac{5}{4} \cdot \frac{1}{3} \cdot p$. Thus, $u(A_1\ \mbox{routes} | B_1\ \mbox{routes}) > u(A_1\ \mbox{hoards} | B_1\ \mbox{routes})$.
\end{proof}

While we can observe that forwarding transactions does modify the probability of producers proposing a block, probability analysis shows that forward-propagation is still statistically dominant. As with our section on UIC, what is really happening is that the impossibility results created by the assumption of "temporary monopoly" are overcome by the use of a work function that explicitly links fee-levels to the pace of block production.

Similar logic shows that fake transactions (producer-initiated fees) are also disincentivized under temporary-monopoly assumptions.

\paragraph{Fake Transactions} 

We consider the case of a block producer who is able to produce a block that solves \ourOPT using at least a subset of the transactions in their mempool. The question is whether this block producer is advantaged by the manipulation of the set transactions in their block. We prove on a case-by-case basis that they are not by showing that there is only one situation in which fee-manipulation can be profitable and then showing that this situation is incentive compatible with \textit{pareto optimality}:

\begin{theorem}\label{thm:MIC}
    Accellerating the burn fee is the only non-losing strategy for producers
\end{theorem}

\begin{proof}

    Let $S^\star$ be the block producer's optimal subset of transactions based on the bids, computed via \ourOPT. The utility to the producer is at most the value of half of the fees in the block minus at minimum the value of the half the fees in the block that originate from the block producer.

    Let $B = \min_{(f, h)\in S^\star} f$ be the minimum accepted transaction.

    \begin{itemize}[leftmargin=*]

        \item \textbf{Case 1.} If the block producer eliminates $B it no longer has adequate routing work to produce a block and has expects reduced income from increased competition for block production. The utility of the block producer is lower with censorship than honest inclusion.

        \item \textbf{Case 2.} If the block producer replaces $B with an identical transaction, its probability of block production is held constant, but it replaces a profitable transaction with one that imposes losses on it personally. The utility of the block producer is lower with this transaction-replacement strategy.

        \item \textbf{Case 3.} If the block producer replaces $B with a self-generated transaction that pays a higher fee, all transactions in the block will pay a higher fee. This is the only potential situation in which the income available to the producer can overcome the losses they impose on themselves by creating a self-generated transaction that has the potential to be profitable. 
    \end{itemize}
\end{proof}

The situation that must be analyzed to determine whether the mechanism is incentive compatible (with pareto optimality) is the third case. But note a fundamental difference between \ourTFM and other TFMs. In this case, by attempting to increase the fees they are able to collect, the block producer is forcing up the burn-fee and pushing the network into a higher-throughput equilibrium that is more secure and more costly to re-organize. The decision to speed-up the blockchain is also tantamount to an increase in the overall supply of blockspace. Supply is expanding to fill an increase in demand. 

Any strategy that accellerates the burn fee involves the block producer \textit{subsidizing} security for the subset of users who have paid a higher fee than $B and who have signalled a preference for faster inclusion at a higher rate. This is by definition a faithful implementation of the mechanism, since the mechanism is imposing a relationship between utility and fee-levels that requires both to move in union. If we take the narrow definition of MIC as \textif{faithful implementation of the mechanism} then our producers are by definition doing that, since they are using their own funds to subsidize the delivery of value to users at fee-levels that users have already indicated is optimal.

\paragraph{Genuine Incentive Compatibility} 

We can move beyond "faithful implementation" and towards full incentive compatibility. To see this, observe that the decision to self-generate a transaction is rational if the block producer earns enough in profit from the transaction fees paid by other users to outweigh the costs they bear from the inclusion of their own fee-bearing transaction. Even in the hypothetical case where generating such a block is profitable, there is an explicit cost that the block producer pays in the form of accepting a lower marginal profitability.

We also note that if producers spend their own money to push up the burn fee, they increase the cost of repeating this strategy, since the cost of this strategy rises as the burn fee rises relative to the networks' volume of organic fee-throughput. Equilibrium must be reached at the point where the losses from self-generation are not larger than the profits available from the other fees included in the block. It follows that any strategic exploitation of Case 3 pushes the network into an equilibrium where supply expands until the marginal profitability of providing more blockspace matches the marginal utility of users purchasing it.

In game theoretic terms, the decision to self-generate is a strategic decision which reveals private information available only to the block producer: information about their marginal profitability given their private cost structure for attracting whatever mixture of public and private fees exists in their mempool. Producers who are efficient at gathering high-fee transactions may choose to self-generate if they fear competition or desire more rapid inclusion. But all shifts of this sort push the network a higher-throughput equilibrium in which the utility delivered to users increases relative to the fees they are paying, and larger losses must be borne to speed up the chain further.

We thus have a functioning market. Users desire transaction inclusion at the lowest rates possible. Producers desire transaction inclusion at the highest rates possibe. The mechanism induces honest revelation from users of the fees they are willing to pay for the utility that the blockchain offers at its equilibrium level as represented by the burn fee. Producers reveal private information about their own cost structure through their decision to generate a transaction that speeds up the blockchain but burns a portion of their own wallet as the cost. Truthful preference revelation and strategic behavior occurs on both sides of the market, and the network reaches an equilibrium at the point where these two forces come into balance, the point of \textit{pareto optimality} at which neither users nor producers can profitably extract more utility from the network by allocating their resources to it in any different proportion.



\section{Compatibility with Pareto Optimality \label{sec::compatibility}}

In the body of this paper, we demonstrated previous impossibility results simply universalize the limitations of \textit{direct mechanisms} (auctions) and their implicit choice of social choice rule. We then introduced an \textit{indirect mechanism} which is not subject to these limitations.

While the previous section shows the preceding impossibility results do not apply to \ourTFM, it does not establish that the mechanism is incentive compatible with the social choice rule of \textit{pareto optimality}. To do that, we must return to economics and discuss how the \ourTFM overcomes the foundational informational impediments to implementing \textit{pareto optimality}: Samuelson's objection based on the existence of public goods, and Hurwicz's objection on the requirements of price-discovery in informationally decentralized mechanisms.

\paragraph{Samuelson and Free-Riding}

Samuelson's objection is based on the two-good equation for the \textit{utility possibilities frontier} which describes all points at which economic production is pareto optimali. His observation was that achieving this equation and thus pareto optimality is problematic in the presence of \textit{private and \textit{public} goods. Rational actors will not keep this equation in equilibrium if they can enjoy the benefits of goods without paying the costs:

\LARGE
\begin{adjustwidth}{1.5em}{1.5em}
\begin{math}
\sum_{i=1}^{s} \frac{u_{{pub}+{priv}}^i}{u_b^i} = \frac{F_{{pub}+{priv}}}{F_b}
\end{math}
\end{adjustwidth}
\normalsize

To solve this problem, observe that transaction inclusion is neither a private good as conceptualized by Roughgarden nor a public good as conceptualized by Fox. The fee paid for blockspace is privately-collected and can be privately-negotiated, but collective security is maximized to the extent its existence induces competition between producers for the right to use the blockchain to collect the fee. This is why transaction hoarding can transform public transactions into private ones by limiting the degree of competition for collection, and why hoarding does not affect the collective security budget in atomistic environments.

\ourTFM skirts this problem through two approaches. The first involves the use of the fee as the direct incentive. This eliminates the ability for participants to increase the relative value of their fee for producers by limiting distribution. Producers who offer participants submarket rates through off-chain payments must add their own fees back to the block in a separate transaction in order to make up for the shortfall in routing work that results. This kills incentives for off-chain payments, or at the least ensures that .

The second technique that eliminates this problem is the explicit incentive of participants to broadcast transactions. We can see how this avoids the problem that Samuelson raised by modifying his cost function and adding a variable \textit{x} that reflects the probabiliy that transactions and fees are circulating publicly, and that open competition thus exists for collection of the transaction fee:

\LARGE
\begin{adjustwidth}{1.5em}{1.5em}
\begin{math}
\sum_{i=1}^{s} \frac{u_{({pub}*{x})+{priv}}^i}{u_b^i} = \frac{F_{{priv}}}{F_b}
\end{math}
\end{adjustwidth}
\normalsize

Ceteris paribus, we know that users prefer widespread distribution of their fee as this maximizes the speed of transaction inclusion. And that producers prefer to have private access to fees in order to free-ride on publicly circulating transactions and improve their relative profitability. Given the proof offered in an earlier section that routing work incentivizes producers to cooperatively share transactions, we can see that these mechanisms avoid the problems Samuelson flagged with suboptimality as the equation for the \textit{utility possibilities frontier} simplifies to the following once \textit{x} becomes 1:

\LARGE
\begin{adjustwidth}{1.5em}{1.5em}
\begin{math}
\sum_{i=1}^{s} \frac{u_{{pub}+{priv}}^i}{u_b^i} = \frac{F_{{priv}}}{F_b}
\end{math}
\end{adjustwidth}
\normalsize

Pareto optimality is achievable in this situation.







\paragraph{Hurwicz and the Incentive to Truthfulness}

The objection that Hurwicz offers to achieving incentive compatibility is based on the informational need for participants to engage in a process of price discovery priour to allocating resources to the purchase or production of utility. This is the source of Hurwicz' distinction between the "pre-exchange negotiation stage" in which participants determine and optimize their resource allocation strategies, and the "action stage" in which they put them into effect. Hurwicz argues that this distinction is always needed as all algorithms capable of optimizing prices over time require users to engage in this form of price discovery.

It is consequently the lack of an "incentive to truthfulness" in this pre-exchange negotiation step that gives participants the ability to strategically mislead others in ways that can lead to suboptimal outcomes. 

The first way in which \ourTFM overcomes this issue is by moving the information needed to calculate prices out of the hands of adversarial peers and into the environment itself. With the cost of transaction inclusion listed directly in the block header, and a cost of chain-extension rooted in real-world hash expenses, calculating the market rate for transaction-inclusion becomes a mathematical exercise that can be performed without the need for off-chain negotiations with peers. Participants can model the expected prices by examining a single block, or through more granulated models that examine the cost of transaction inclusion over time.

But don't users need to get this information about the blockchain from their peers? While it might seem that block producers can forge the information as a form of strategic manipulation, the ability of the mechanism to prevent the costless inclusion of self-generated transactions imposes an asymmetrical cost on the communication of fraudulent information.

Abstractly, the market price is the competitive price at which all transactions which pay the market rate are included, and no transactions which do not pay the market rate *are* included. The ability to create a mechanism that punishes transaction exclusion thus implies the ability to create a mechanism that makes pushing the equilibrium price of transactions away from the pareto optimal point.

We observe that this \ourTFM passes the criterion offered by Hurwicz for algorithmic compatibility with a pareto optimal process. The presence of a historical blockchain allows for optimization processes that involve \textit{inertia}, while there is an inherent \textit{incentive to truthfulness} in the informational exchanges that participants may need to may to engage in the process, such as the sharing of the block headers that contain pricing information. As an additional note, we note that the presence of smoothed payouts allows for smoothed price-adjustments that overcome the objections of critics like Jordan (1986) to the theoretical attainment of pareto optimality itself.





\section{Conclusion \& Future Work \label{sec::conclusion}}

In this paper, we introduced \textsc{RTR-TFM}: a novel TFM that addresses the incentive misalignment in classic transaction fee mechanisms (TFMs) by introducing a novel routing-based block production rule and a revenue scheme. \ourTFM\ rewards block producers in proportion to their contribution to the propagation of transactions. Such a reward ensures that block producers actively participate in the blockchain network upkeep instead of free-riding on other participating nodes. We also provide a game-theoretic characterization of the underlying game in \ourTFM. We prove that \textsc{RTR-TFM} effectively discourages transaction hoarding, ensures Sybil resistance, and achieves incentive compatibility for both users and block producers under reasonable assumptions.

In addition to demonstrating these properties on the technical level, we also show that \ourTFM addresses the underlying informational problems that create the forms of abstract \textit{goal conflict} that lead rational actors to launch byzantine attacks on \textit{TFMs}, creating a new kind of informational environment more conducive to the implementation of \textit{pareto optimality} as a social choice rule in distributed mechanisms.





%%%%%%%%%%%%%%%%%%%%%%%%%%%%%%%%%%%%%%%%%%%%%%%%%%%%%%%%%%%%%%%%%%%%%%%%

%%% The next two lines define, first, the bibliography style to be 
%%% applied, and, second, the bibliography file to be used.
\cleardoublepage
\bibliographystyle{ACM-Reference-Format} 
\bibliography{ref}

%%%%%%%%%%%%%%%%%%%%%%%%%%%%%%%%%%%%%%%%%%%%%%%%%%%%%%%%%%%%%%%%%%%%%%%%

\end{document}

%%%%%%%%%%%%%%%%%%%%%%%%%%%%%%%%%%%%%%%%%%%%%%%%%%%%%%%%%%%%%%%%%%%%%%%%


