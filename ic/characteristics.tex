
The novel characteristics of \textit{TFMs} that lead to suboptimal provision are \emph{non-excludability}, \emph{self-provision} and \emph{informational decentralization}. 

\paragraph{Non-Excludability} allows anyone to use or provision these networks on equal terms provided they are willing to pay a competitive market price. This economic characteristic underpins the technical properties of \textit{censorship resistance}, \textit{decentralization} and \textit{network resilience}: censorship requires a mechanism with the power to exclude; centralization implies barriers to entry; resilience comes from the ability to route around byzantine nodes by adding new ones to the network. Non-excludability also contributes to economic efficiency in \textit{TFMs}, as efficiency is maximized when producers build atop blocks proposed by their peers rather than orphaning them.

\paragraph{Informational Decentralization} is not the casual concept of \textit{decentralization} as used in computer science (see: non-excludability) but the economic definition offered by Hurwicz (1972) for mechanisms in which "participants have direct information only about themselves." This characteristic makes \textit{TFMs} vulnerable to the attacks, identified by Hurwicz, in which participants manipulate the informational environment that others rely on to make strategic decisions.

\paragraph{Self-Provision} allow \textit{TFMs} to support themselves without an owner, relying instead on payouts to network participants. While volunteer-run networks are theoretically possible, their designs fall outside the scope of \textit{TFMs} as transaction fees are purely redistributive. For this reason, in volunteer mechanisms the imposition of fees leads to a dead-weight efficiency loss, since any fee-level above zero is strictly suboptimal given the cost structure of the network.

These three characteristics create fundamental tensions that \textit{TFMs} struggle to reconcile. Mechanisms must permit open access without enabling sybil attacks, offer private benefits without socializing losses, and use decomposable algorithms while resisting byzantine attacks on the message-passing layer of the information environment. We can see the importance of all three characteristics from the way they form an \emph{economic trilemma} where the removal of any one property offers immediate relief to the problems created by the other two.

Understanding these characteristics allows us to identify the specific types of \textit{goal conflict} that motivate for-profit attacks on \textit{TFMs}. The first type, conflict rooted in \textit{self-interest}, occurs when participants prefer to allocate their resources differently than the mechanism designer intends, such as to purchase a different combination of goods and services across the economy as a whole. For an example of this, a user might desire to save a portion of their transaction fee to spend on a cheaper form of data transfer instead of offering a fee at the mechanism's optimal allocation. In this case, participants are signalling disagreement with the designer's intended allocation of utility, both \textit{within} the mechanism and \textit{between} the mechanism and other external goods. These attacks consequently involve participants choosing to bid at suboptimal fee levels, as they prioritize their personal preferences over the collective optimal outcome.

The second form of goal-conflict observable in \textit{TFMs} is \textit{free-riding}, which emerges when the characteristics of \textit{non-excludability} and \textit{self-provision} combine to create public goods within the consensus mechanism. While free-rider pressures are common in many mechanisms, in \textit{TFMs} they are particularly intractable due to the presence of two-sided free-rider problem where users and producers can free-ride on the mechanism in different ways: producers by maximizing the revenue they extract from any collective payout like the block reward, and users by minimizing their contribution to the security budget. As our next section explains, these are the class of attacks that manifest in the form of side-contract payments, while is why attempts to mitigate other forms of goal conflict fail to make headway disincentivizing collusion.

Our third form of goal-conflict is \textit{strategic manipulation}, which emerges because -- as Leonid Hurwicz observed -- in informationally-decentralized mechanisms participants can strategically manipulate others into suboptimally allocating their own resources by manipulating the informational space in which they form their expectations of their strategic environments. This type of goal-conflict is what incentivizes producers to create fake transactions and fake blocks, and what incentivizes users and producers to exploit threshold vulnerabilities in auction designs. This is the main problem mechanism designers have traditionally sought to eliminate through the use of techniques that attempt to achieve bayesian incentive compatibility or otherwise incentivize the truthful revelation of preferences.

As should be obvious, disagreements motivated by \textit{self-interest}, \textit{free-riding}, and \textit{strategic manipulation} are fundamentally distinct types of goal conflict. The source of the first is psychological: in the private perception of the individual that their utility is better maximized through a different resource allocation strategy. The source of the second is in the nature of the utility-providing good, whose indivisibility encourages participants to minimize their own contributions to its provision. And the source of the third is in the informational environment of the market itself, in which the costless ability for participants to mislead others can be manipulated by rational actors to induce others into forming strategies that misallocate their resources to the benefit the manipulating party.

The fact that conflicts motivated by self-interest, free-riding and strategic manipulation constitute distinct types of goal-conflict is also why each type of attack within \textit{TMFs} is expressed in different ways and through unique social dynamics. Conflicts motivated by considerations of self-interest are expressed through unilateral changes to the fees offered for transaction inclusion by the fee-paying user. Conflicts motivated by an incentive to free-ride require cooperative attempts to defund the mechanism by a subset of network participants, since at least one producer must team up with at least one user in order to enable either to free-ride on the mechanism. Conflicts motivated by strategic manipulation are adversarial and primarily involve price-manipulation strategies such as bid-shading by users or the costless inclusion of transactions that manipulate the fee-level by producers.

The fundamentally different source of the motivations to attack \textit{TFMs} and the different ways in which these attacks manifest is the primary reason that achieving fee-optimality seems like such an intractable problem. Any general solution requires the \textit{TFM} to implement an equilibrium in which none of these conflicts exist, which requires finding a technical solution that eliminates three fundamentally different motivations to byzantine behavior which can be expressed through unilateral, cooperative or adversarial strategies. It is little surprise that the existing literature has , especially in the face of auction designs that are primarily intended to addres only the third problem.

But there is a solution, which is why in our next section we pull back to economic theory to show why \textit{pareto optimality} must be the social choice rule chosen by mechanism that seeks optimal fee-throughput in a collusion-free equilibrium, as it is the only known equilibrium in which all three forms of \textit{goal conflict} do not exist.

