
The novel characteristics of \textit{TFMs} that lead to suboptimal provision are \emph{non-excludability}, \emph{self-provision} and \emph{informational decentralization}. 

\paragraph{Non-Excludability} allows anyone to use or provision the networks on equal terms provided they are willing to pay the competitive market price. This economic characteristic underpins the technical properties of \textit{censorship resistance}, \textit{decentralization} and \textit{network resilience}: censorship requires a mechanism with the power to exclude; centralization creates barriers to entry; resilience comes from the ability to route around byzantine actors by adding new nodes to the network. Non-excludability also contributes to economic efficiency in \textit{TFMs}, as efficiency is maximized when producers build atop blocks proposed by their peers rather than orphaning them.

\paragraph{Informational Decentralization} refers not to the casual concept of \textit{decentralization} as used in computer science (see: non-excludability) but the economic definition offered by Hurwicz (1972) for mechanisms in which "participants have direct information only about themselves." This characteristic makes \textit{TFMs} vulnerable to Byzantine strategies, as identified by Hurwicz, in which participants manipulate the informational environment that others rely on to make strategic decisions.

\paragraph{Self-Provision} allow \textit{TFMs} to support themselves without an owner, relying instead on payouts to network participants. While volunteer-run networks are theoretically possible, their designs fall outside the scope of \textit{TFMs} as transaction fees are purely redistributive. For this reason, in volunteer mechanisms the imposition of fees leads to a dead-weight efficiency loss, since any fee-level above zero is strictly suboptimal given the cost structure of the network.

These three characteristics create fundamental tensions that \textit{TFMs} struggle to reconcile. They must permit open access without enabling sybil attacks, offer private benefits without socializing losses, and use decomposable algorithms while resisting byzantine manipulation of the information environment. We can see the importance of all three characteristics from the way they form an \emph{economic trilemma} where the removal of any one property offers immediate relief to the problems created by the other two.

% Whereas non-excludability allowed anyone to enter and benefit from the system, with exclusionary rules the system can more easily assign costs and rewards to aligns payouts with individual contributions, preventing byzantine attacks on the payout mechanism and other inefficiencies. Non-excludability combined with informational decentralization makes it difficult to manage the information flow within the system. By eliminating non-excludability, we can impose centralized control over information flow or restrict the ways in which participants broadcast messages, thereby simplifying the design of decomposable algorithms to ensure that participants can’t easily manipulate the signals used to determine transaction fees. In theoretical terms, the elimination of non-excludability is equilvalent to the privatization of a public commons: the problem of unrestrained value-extraction is eliminated by the insertion of a party who is motivated and empowered to maximize the productive capacity of the commons.
% 
% Similarly, eliminating self-provision restores the network to a volunteer-driven model. This reduces the incentive for network participants to engage in for-profit exclusionary tactics like orphaning blocks or pushing competitors off the network. Because volunteer provision relies on participants contributing without direct financial incentives, it also reduces the motivation for participants to manipulate the informational environment. Eliminating self-provision is consquently a popular compromise among blockchain developers, and the entire Bitcoin scaling war can be viewed as an ideological debate over whether commercial provision of a proof-of-work network at scale necessarily undermines the desirable properties of "decentralization" (see: non-excludability).
% 
% Informational decentralization is the most difficult characteristic to sacrifice. However, eliminating it allows for the achievement of any economically viable equilibrium by fiat, as it is only the strategic choices of a single actor that determines whether the mechanism implements any particular social choice rule. With informational centralization the process of making decisions about optimal resource allocation also becomes easier, since the centralized mechanism can directly observe the true state of the network.

Understanding these characteristics allows us to identify the three types of \textit{goal conflict} that drive byzantine attacks on \textit{TFMs}. The first type, conflict rooted in \textit{self-interest}, occurs when participants prefer to allocate their resources differently than the mechanism designer intends. For example, a user might desire to save a portion of their transaction fee to spend on other goods and services, rather than adhering to the mechanism's optimal allocation. In this case, participants are signalling disagreement with the designer's intended allocation of utility, both \textit{within} the mechanism and \textit{between} the mechanism and other external goods. These attacks consequently involve participants choosing to bid at suboptimal fee levels, as they prioritize their personal preferences over the collective optimal outcome.

Our second form of goal-conflict is \textit{free-riding}, which emerges because the combination of \textit{non-excludability} and \textit{self-provision} creates public goods within the consensus mechanism. While free-rider pressures are common in many mechanisms, in \textit{TFMs} they are more intractable due to the presence of a dual-sided free-rider problem where users and producers can free-ride on the mechanism in different ways: producers by maximizing the revenue they extract from any collective payout like the block reward, and users by minimizing their contribution to the security budget. As our next section explains, these are the class of attacks that manifest in the form of side-contract payments.

Our third form of goal-conflict is \textit{strategic manipulation}, which emerges because -- as Leonid Hurwicz observed -- in informationally-decentralized mechanisms participants can strategically manipulate others into suboptimally allocating their own resources by manipulating the informational space in which they form their own strategies. This class of goal-conflict incentivizes producers to create fake transactions, and users to exploit threshold vulnerabilities in auction designs. It is the main problem mechanism designers eliminate when they design mechanisms that achieve bayesian incentive compatibility or incentivize the truthful revelation of preferences.

While conflict over self-interest, free-riding and strategic manipulation are all distinct types of goal-conflict, each type has different causes and manifests in different ways. This is the reason incentive compatibility seems so intractable in \textit{TFMs}, as techniques intended to prevent \textit{strategic misrepresentation} cannot eliminate goal conflict entirely unless conflicts over self-interest and free-riding are also addressed. Any general solution requires the \textit{TFM} to implement an equilibrium in which none of these conflicts exist, which is why the next section pulls back to economic theory to show why \textit{pareto optimality} must be the social choice rule chosen by mechanism that seeks optimal fee-throughput in a collusion-free equilibrium.

