
\emph{Transaction Fee Mechanisms} (TFMs) refer to a class of distributed systems in which a consensus mechanism governs the allocation of the same resource that it uses to incentivize its own provision. Unlike traditional mechanisms, where the number of honest and dishonest processes is static, in TFMs voting power is dynamic — it flows with the payouts issued by the mechanism. This introduces the ability for attacks intended to extract profits from this circular flow of funds to compromise the stability of the mechanism and subvert its ability to sustain itself in an optimal equilibria.

As more and more of these attacks have been identified in the wild, academics have typically named them after the "mechanism-specific" techniques they exploit, resulting in a wide array of terminology such as sybil attacks, block-orphaning attacks, selfish mining attacks, fee manipulation attacks, eclipse attacks, side-contract payments, and others. While most researchers treat these vulnerabilities as isolated technical challenges, a few scholars have applied concepts from economics and particularly mechanism design to ask whether general solutions are possible. Unfortunately, this has led to a series of impossibility results that suggest designing socially optimal TFMs may be infeasible.

This paper challenges these results by identifying the exact equilibrium in which all such attacks are irrational. It argues that three distinct types of goal conflict -— self-interest, free-riding, and strategic manipulation —- are what prevent this equilibrium from being implemented by most \textit{TFMs}. We then review the earlier work in the field to show why the problem seems insolvable: a methodological reliance on direct mechanisms and specifically auction models limits the ability of the field to address all three types of goal conflict or even handle the informational complexity necessary to compute the required equilibria in which none apply.

Using the language of mechanism design, this paper demonstrates that the social choice rule needed to achieve fee-optimality and collusion-resilience is \textit{pareto optimality}, but the direct mechanisms used to model TFMs are incapable of implementing this rule, as doing so requires multi-dimensional preference revelation across a high-dimensional preference space -- a level of informational complexity that composable algorithms cannot handle. While Maskin's Revelation Principle teaches that a direct mechanism must exist for any indirect mechanism, in this case achieving optimality requires decomposable algorithms that use the "no-trade option" to reduce the complexity of computation and limit the scope of the state transitions proposed to those consistent with an efficiency shift towards \textit{pareto optimality}.

Since familiarity with economics is needed to understand what our three types of goal conflict are and why they cannot be eliminated by the kind of direct mechanisms preferred by the field, the next section of this paper begins by identifying the novel characteristics of TFMs, and showing how they create problems with self-interest, free-riding and strategic manipulation. We then show why \textit{pareto optimality} is the social choice rule needed to eliminate all three, which leads to a review of the impossibility results mentioned above and a demonstration that the conclusions of these papers reflect the informational limitations of the models used to analyse the problem.

In the second half of this paper, we introduce a novel class of indirect mechanism that solves this problem and is theoretically compatible with \textit{pareto optimality}. We provide the formula for this mechanism and then a game-theoretic treatment of it which proves its inconsistency with the impossibility results discussed earlier. We then close with a return to economic theory and discussion of how the mechanism overcomes the more foundational theoretical problems identified by Samuelson and Hurwicz in the last century to the design of mechanisms that implement pareto optimality as a social choice rule.

