
In the field of economics, the pioneering work on welfare optimality was the publication of Vilfredo Pareto’s "Cours d'économie politique" (1896), which introduced the concept of pareto optimality. Pareto defined this state as one where resources are allocated so efficiently that it is impossible to improve overall social welfare by changing the way in which resources are allocated to the production of utility.

From a mathematical perspective, Pareto optimality is achieved when the marginal utility derived from the last unit of each good purchased by each individual is proportional to its production cost. This implies that individuals are spending their resources in a way that maximizes their utility -— essentially, every dollar is spent on whatever good or service provides the greatest marginal benefit to its consumer. This allocation is considered individually rational and provides two important social critera demanded by \textit{TFMs}. First, mechanisms in a \textit{pareto optimality} equilibrium are free from conflicts involving self-interest since no party will unilaterally desire to pay a greater or lesser fee. Second, \textit{pareto optimality} has attractive collusion-proof properties: if no individual can reallocate his own resources without making himself worse off, no group of similar individuals can collude to do so without at least one member of the group suffering as a result. This eliminates all categories of user-user and producer-producer collusion.

But is it possible for \textit{pareto optimal} equillibria to be robust against goal-conflict involving \textit{free riding} pressures or \textit{strategic manipulation}?

The first question was addressed by Samuelson (1954) when he observed that achieving optimal production levels is challenging for goods with non-excludable benefits. If users can lie about the utility they receive from such goods they can pay a lower fee themselves while enjoying the higher level of utility funded by contributions from their honest peers. It was Samuelson's demonstration of this problem -- that individual rationality subverted pareto optimality -- that led \citet{hurwicz1960optimality} to coin the term \textit{incentive compatibility} in reference to the opposite condition, the state in which the utility-maximizing behavior of individuals is \textit{compatible with} or leads emergently to the desired welfare condition referred to as \textit{social choice rule}.

Samuelson's observation is why free-rider pressures constitute the second type of \textit{goal conflict} within \textit{TFMs}, where they manifest in the form of side-contract payments. From the perspective of users, selling transactions to block producers gives producers the ability to collect their fees without the need to compete so intently for the privilege. Producers will happily accept a lower fee from users as less of their own income need flow into the public security budget. This form of collusion involves producers helping users free-ride on the contributions of other users to the collective security budget, as analogous to the classic free-rider in Samuelson's model.

On the producer side, side-contract payments permit block producers to free-ride on their peers as well. In this second case, producers offer users transaction-inclusion at suboptimal rates because private control of the transaction fee expands the producer's share of blocks committed to the longest-chain, allowing them to extract more income from any non-excludable payout like the block reward than they lose by subsidizing the user's transaction. Once again we have a situation analogous to Samuelson's model, except in this case the incentive to collude comes from producers and the incentive is to collect more in revenue not pay less in fees.

Understanding the two-sided nature of free-riding in \textit{TFMs} is critical for designing mechanisms that eliminate this form of goal conflict. In the absence of this understanding, it is common to consider all forms of user-producer agreement as suboptimal. But this is not the case! If price negotiations between users and producers drive the cost of blockspace towards \textit{pareto optimal} levels without affecting the overall level of public good provision, they technically shift the network into a more efficient equilibrium in which fee-throughput level are more optimal and \textit{goal conflict} is avoided. It is also trivial to see that side-contract payments will never drive transaction fees below the cost of blockspace in the absence of public goods, as rational producers cannot sustainably accept transaction fees that are lower than their private cost of providing blockspace. 

The inability of proof-of-work and proof-of-stake designs to contain fundamental pressures to free-ride is a major cause of inefficiency and suboptimality in those networks. As we shall see, these pressures are also responsible for a non-trivial number of impossibility results, since the techniques mechanism designers use to prevent other classes of goal conflict -- such as inducing truthful preference revelation -- can contain adversarial forms of strategic manipulation but fail to prevent the sorts of cooperative attacks we see with free-riding strategies.

Our first two classes of goal conflict are thus "self-interest" and "free-rider pressures". The first exists in mechanisms that lack \textit{pareto optimality} and can be solved only by designing mechanisms that implement that social choice rule. The second subverts the ability of mechanisms to achieve \textit{pareto optimality} and can only be rectified by eliminating the public goods that lurk within their incentive sub-structures.

This leaves our third category of \textit{goal conflict}, which is the practice of \textit{strategic manipulation}. To put this issue in historical context, it is useful to know that by the late 1950s and 1960s, the problems that Samuelson flagged regarding the efficient provision of public goods had become widely accepted in mainstream economics. Nonetheless, most economists still believed the production and trade of private goods under classical assumptions was more-or-less \textit{pareto optimal}. Or so they believed until 1972 when \citet{hurwicz1973design}, in his second great contribution to mechanism design, pointed out that similar problems also subvert the \textit{pareto optimal} provision of private goods in informationally decentralized mechanisms. 

The cause of the suboptimality Hurwicz identified came from the need for participants to exchange information as part of their price-discovery process. In any situation where agents could manipulate the informational environment they could theoretically induce others to strategically misallocate their own resources. The particular passage in Hurwicz's paper that points this out is worth quoting in full:

\begin{quote}
Economists have long been alerted to this issue by Samuelson (1954) in the context of the allocation problem for public goods. But, in fact, a similar problem arises in a "nonatomistic" world of pure exchange of exclusively private goods.... If [two parties] were both told to behave as price-takers it would pay one of them to violate this rule if he could get away with it. Now we assume that he cannot violate the rule openly, but he can "pretend" to have preferences different from his true ones. The question is whether he could think up for himself a false (but convex and monotone) preference map which would be more advantageous for him than his true one, assuming that he will follow the rules of price-taking according to the false map while the other trader plays the game honestly. It is easily shown that the answer is in the affirmative. Thus, in such a situation, the rules of perfect competition are not incentive-compatible.
\end{quote}

In this case, our form of goal conflict does not involve participants re-allocating their own resources (self-interest) or cooperating with others to exploit public goods (free-riding) but adversarially manipulating the informational environment to frustrate efficient price-discovery. In the context of \textit{TFMs}, we see this exploited whenever producers put their own fees into blocks, or costlessly loop money around the chain.

Awareness of these informational attacks is what led Hurwicz to develop his framework for studying \textit{incentive compatibility}, which asks whether specific market structures (mechanisms) can achieve (implement) specific outcomes (social choice rules) in the presence of participants who make strategic decisions on the basis of private information. This is the reason "truthful revelation of preference" is considered such an attractive property in mechanism design, as it implies the mechanism is not vulnerable to this particularly category of goal conflict.

As an aside, since several papers on \textit{TFMs} declare \textit{incentive compatibility} impossible to achieve, it is useful to remember that Hurwicz never made this claim. As his student Eric Maskin later pointed out, such claims show a misunderstanding of the framework, since all mechanisms are by definition incentive compatible with their outcomes. What a failure of incentive compatibility means is that if private preferences are used to form the strategies adopted by participants in a mechanism, then without an "incentive for truthfulness" mechanisms cannot be assumed capable of implementing any social choice rule. 

This point is important for ultimately implementing {pareto optimality} within a distributed system. For while Hurwicz is often misinterpreted as implying that the direct revelation of preferences is a pre-condition for achieving \textit{pareto optimality}, the truth is more nuanced -- market structures still exist which lack the problems Hurwicz identified with \textit{strategic manipulation}, the key exceptions being \textit{atomistic} markets characterized by perfect competition, markets in which the utility purchased varies with price paid, and markets lacking a pre-exchange messaging step. Eric Maskin, who later win the Nobel Prize for his work on the revelation principle, confirmed Hurwicz's intuition when he found that \textit{pareto optimality} is possible in some market structures without the need for truthful preference revelation as an intermediary step.\citet{maskin1999nash}. His revelation principle also illustrates this in a more subtle way, by showing that a symmetry of outcomes must exist between mechanisms where information is computed in decomposable fashion using agent-level functions, and mechanisms where the exact same information is revealed truthfully and the computation is performed by a centralized mechanism in a non-decomposable fashion. As Maskin showed, if the centrally-computed outcome does not result in a Nash Equilibrium then the decomposable function cannot have one and at least one agent must be lying about their true preferences.

Maskin's work revealed a deeper truth: all incentive compatible mechanisms will induce the revelation of private information one way or the other, meaning that the difference between mechanisms is not whether they reveal user preferences so much as whether they reveal those preferences \textit{directly} or \textit{indirectly}. In direct mechanisms participants share their preferences truthfully in the pre-exchange negotiation step, while in \textit{indirect} mechanisms they reveal them either obliquely in the price-discovery process (such as by negotiating for bundles of goods) or by skipping the price-discovery stage and simply submitting purchase orders directly onto the market.

Back on topic, since \textit{TFMs} are \textit{informationally decentralized} mechanisms that involve users and producers making strategic decisions on the basis different preferences for the allocation of resources within the mechanism, if our social choice rule is \textit{pareto optimality}, we cannot achieve it in any mechanism where participants can costlessly mislead others by manipulating any information relevant to fee-levels in the state of consensus. If a mechanism permits block producers to costlessly include their own transactions in blocks we thus have \textit{de facto} grounds for concluding that incentive compatibility with \textit{pareto optimal} fee-throughput will be impossible to achieve in that mechanism. Strategic manipulation can only be eliminated in mechanisms that make the inclusion of self-generated transactions costly, such that the decision by a block producer or user to include their own fees in a block reveals private information that the mechanism can exploit to shift its own provision into a more efficient equilibrium.

Hurwicz (1973) provides several other conditions any \textit{TFM} will need to meet in order to successfully implement \textit{pareto optimality}. The first is that one-shot mechanisms are insufficient, since algorithms with \textit{inertia} are required to iterate price levels into their optimal positions over time [CITE]. This suggests that the information required to calculate the price of blockspace must be somehow calculable from the state of consensus rather than collected exclusively from peers. And as Jordan (1986) observes, some form of smoothing of costs or payouts is beneficial to prevent mechanisms from unpredictably oscillating around the desired equilibrium point. As we shall see in the next section, these requirements are also incompatible with the vast majority of papers attempting to model the feasibility of building a dream TFM. 

In summary, our three types of goal conflict -- self-interest, free-riding, and strategic manipulation -- are distinct issues that affect most \textit{TFMs}. All three undermine the ability of any mechanism to achieve \textit{pareto optimality} which in turn prevents them from targeting a highly efficient and collusion-proof equilibrium. Each type of goal conflict manifests as unique technical attacks involving different actors, different types of messaging, and targeting different steps in the operation of the consensus mechanism. A block producer who floods the network with spam transactions to drive up fees is engaging in strategic manipulation. A threshold user who underbids in a Vickrey-Clarke-Groves auction is exhibiting self-interest. Users who conspire with producers to defund the security budget are free-riding on their non-colluding counterparts. All three classes must be eliminated to achieve an optimal \textit{TFM}, which is why is why achieving it is so difficult in practice.

With this framework in place for understanding the categories of problems \textit{TFMs} face, in the following section we turn our attention to the existing literature on \textit{TFMs} in computer science, with the goal of showing why the impossibility results in these papers reflect the limitations of their models rather than the limits of what is possible in distributed systems.

