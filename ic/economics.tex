
Eliminating all forms of goal conflict is possible in any equilibrium that pushes production onto the \textit{utility possibilities frontier}, a curve that defines the maximum amount of utility that can be produced given the rational allocation of resources belonging to participants in the system. The relevant subfields in economics that describe this challenge deal with welfare optimality, public choice theory, informational economics and mechanism design.

In the case of welfare economics, the pioneering work on optimality was the publication of Vilfredo Pareto’s "Cours d'économie politique" (1896), which introduced the concept of pareto optimality. Pareto defined this state as one where resources are allocated so efficiently that it is impossible to improve overall social welfare by changing the way in which resources are allocated to the production of utility. Pareto optimality is thus the term used to describe any state of production that exists on the \textit{utility possibilities frontier}.

From a mathematical perspective, pareto optimality is achieved when the marginal utility derived from the last unit of each good purchased by each individual is proportional to its production cost. This implies that individuals are allocating their resources so as to maximize their utility -— every dollar is spent on whatever good or service provides the greatest marginal benefit. This allocation is considered individually rational and provides two important social critera demanded by \textit{TFMs}. First, it frees mechanisms from conflicts involving self-interest since no party will unilaterally desire to pay a greater or lesser fee. Second, \textit{pareto optimality} has attractive collusion-proof properties: if no individual can reallocate his resources without making himself worse off, no group of individuals can collude to do so without at least one member of the group suffering as a result. This eliminates all categories of user-user and producer-producer collusion.

While a \textit{pareto optimal} equilibrium eliminates all forms of goal conflict motivated by \textit{self-interest}, can such an equillibrium can be sustained in the face of \textit{free riding} pressures or \textit{strategic manipulation}?

The first question was asked by Samuelson (1954) when he observed that achieving \textit{pareto optimality} is difficult for goods with non-excludable benefits. If users can misallocate their own resources while enjoying the non-excludable forms of utility funded by contributions from their honest peers. It was Samuelson's mathematical demonstration of this problem -- that individual rationality subverted pareto optimality -- that led to the emergence of public choice theory, and prompted \citet{hurwicz1960optimality} to coin the term \textit{incentive compatibility} in reference to the opposite condition, the state in which the utility-maximizing behavior of individuals is \textit{compatible with} or leads emergently to the desired welfare condition referred to as its \textit{social choice rule}.

\LARGE
\begin{adjustwidth}{1.5em}{1.5em}
\begin{math}
\sum_{i=1}^{s} \frac{u_a^i}{u_b^i} = \frac{F_a}{F_b}
\end{math}
\end{adjustwidth}
\normalsize

Samuelson's observation -- expressed in the equation above for the \textit{utility possibilities frontier} in the generalizable two-good model -- explains why free-rider pressures subvert \textit{pareto optimality} in \textit{TFMs}. Achieving \textit{pareto optimality} requires users to allocate resources in whatever proportion keeps their costs in alignment with the utility they enjoy, yet the existence of non-excludable benefits creates rational pressures to defect. In \textit{TFMs} this manifests in the form of side-contract payments and other strategies that restrict competition for monetization of the fee. From the perspective of users, selling transactions gives producers the right to collect fees without the need to compete so intently for the privilege. Producers happily accept a lower fee as less of their own income needs flow into the collective security function as the cost of collection. This form of collusion is analogous to the classic free-rider in Samuelson's model.

On the producer side, side-contract payments permit block producers to free-ride on their peers as well. In this second case, producers offer users transaction-inclusion at suboptimal rates because private control of the transaction fee expands their share of blocks committed to the longest-chain, allowing them to extract more income from any non-excludable payout like the block reward. Once again we have a situation analogous to Samuelson's model, except in the motivation is the incentive to collect more in revenue rather than pay less in fees.

Understanding the two-sided nature of free-riding in \textit{TFMs} is critical for understanding why achieving \textit{pareto optimality} is so challenging. In the absence of this understanding, it is common to hear all forms of user-producer agreement described as suboptimal. But this is not the case! If price negotiations between users and producers drive the cost of blockspace towards \textit{pareto optimal} levels without reducing the overall funding for public good provision, they technically shift the network into a more efficient equilibrium in which fee-throughput levels are \textit{pareto optimal} and \textit{goal conflict} motivated by \textit{self-interest} is minimized. It is also trivial to see that side-contract payments can never drive transaction fees below the cost of blockspace in the absence of public goods, as rational producers cannot sustainably accept transaction fees that are lower than their cost of provision.

The inability of proof-of-work and proof-of-stake designs to contain fundamental pressures to free-ride is a major cause of suboptimality in those designs. As we shall see in our next section, these pressures are also responsible for a non-trivial number of impossibility results, since the techniques mechanism designers use to prevent \textit{strategic manipulation} -- pricing mechanisms intended to induce truthful preference revelation -- can contain those adversarial strategies but not the sort of co-operative attacks we see expressed through free-riding strategies.

The connections betwen our first two classes of \textit{goal conflict} and \textit{pareto optimality} are now clear. Conflict motivated by \textit{self-interest} exists in mechanisms that lack \textit{pareto optimality} and can be solved only by designing mechanisms that implement that social choice rule. The existence of \textit{free riding} pressures within any mechanism subverts the ability of those mechanism to sustain \textit{pareto optimality} by creating incentives for defection even when utility is being produced at optimal levels. This leaves our third category of \textit{goal conflict}, \textit{strategic manipulation}, which falls in the domain of informational economics and mechanism design. 

To put these subfields in historical context, it is useful to know that by the late 1950s and 1960s, the problems that Samuelson flagged regarding the efficient provision of public goods had become widely accepted in mainstream economics. Nonetheless, most economists still believed the production and trade of private goods under classical assumptions was more-or-less \textit{pareto optimal}. Or so they believed until 1972 when \citet{hurwicz1973design}, in his second great contribution to mechanism design, pointed out that similar problems could subvert the \textit{pareto optimal} provision of private goods in informationally decentralized mechanism.

The cause of sub-optimality that Hurwicz identified came from his study of the algorithms that other economists proposed to explain how prices move towards optimality over time. Specifically, what Hurwicz saw was that a "pre-exchange messaging step" existed in all informationally decentralized algorithms as message-passing was an essential prerequisite to the computing of expected market prices, and that the computing of this information was needed for participants to develop their welfare-maximizing strategies. As a result, in any situation where agents could costlessly manipulate price expectations they could theoretically induce others to strategically misallocate their own resources and frustrate the ability of any welfare-maximizing algorithm to approximate its intended outcome. The particular passage in Hurwicz's paper that points this out is worth quoting in full:

\begin{quote}
Economists have long been alerted to this issue by Samuelson (1954) in the context of the allocation problem for public goods. But, in fact, a similar problem arises in a "nonatomistic" world of pure exchange of exclusively private goods.... If [two parties] were both told to behave as price-takers it would pay one of them to violate this rule if he could get away with it. Now we assume that he cannot violate the rule openly, but he can "pretend" to have preferences different from his true ones. The question is whether he could think up for himself a false (but convex and monotone) preference map which would be more advantageous for him than his true one, assuming that he will follow the rules of price-taking according to the false map while the other trader plays the game honestly. It is easily shown that the answer is in the affirmative. Thus, in such a situation, the rules of perfect competition are not incentive-compatible.
\end{quote}

In this case, our form of \textit{goal conflict} does not involve participants re-allocating their own resources (\textit{self-interest}) or co-operating with others to underfund public goods (\textit{free-riding}) but involves participants adversarially and \textit{strategically manipulating} the informational environment to frustrate efficient price-discovery and induce others to misallocate their own resources. In the context of \textit{TFMs}, we see this exploited whenever producers put their own fees into blocks, whenever users engage in bid-shading, or whenever any participants costlessly loop money around the chain to create fraudulent representations of blockchain history.

Awareness of these informational manipulation is what led Hurwicz to develop his framework for studying \textit{incentive compatibility} and launch the subfield of mechanism design, which uses mathematical logic to study whether specific market structures (mechanisms) can achieve (implement) specific outcomes (social choice rules) in the presence of participants who make strategic decisions on the basis of private information. The informational nature of the problem -- caused by the assumed ability of participants to costlessly distort market perceptions -- is the reason "truthful revelation of preference" is considered such an attractive property in mechanism design, as it implies the mechanism is not vulnerable to this particularly category of goal conflict.

As an aside, since several papers on \textit{TFMs} declare \textit{incentive compatibility} impossible to achieve, it is useful to remember that Hurwicz never made this claim. As Eric Maskin later pointed out, such claims show a misunderstanding of Hurwicz' framework, since all mechanisms are by definition incentive compatible with their outcomes. What a failure of incentive compatibility means is that if private information is used as an input to form the strategies adopted by participants in any mechanism, then without an "incentive for truthfulness" those mechanisms cannot be assumed capable of implementing any social choice rule.

This carries back us to our discussion of the challenge of implementing {pareto optimality} within \textit{TFMs}. For while Hurwicz is often misinterpreted as implying that the direct revelation of preferences is a pre-condition for achieving \textit{pareto optimality}, the truth is more nuanced -- market structures still exist which lack the problems Hurwicz identified with \textit{strategic manipulation}, the key exceptions being \textit{atomistic} markets characterized by perfect competition, markets in which the utility purchased varies with price paid, and markets lacking a pre-exchange messaging step. Eric Maskin, who would later win the Nobel Prize for his work on the revelation principle, confirmed Hurwicz's intuition when he found that \textit{pareto optimality} is possible in some market structures without the need for truthful preference revelation as an intermediary step.\citet{maskin1999nash}. His revelation principle also confirms this in a more subtle way, by showing that a symmetry of outcomes must exist between mechanisms where information is computed in decomposable fashion using agent-level functions, and mechanisms where the exact same information is revealed truthfully and the computation is performed by a centralized mechanism in a non-decomposable fashion. As Maskin showed, if the centrally-computed outcome does not result in a Nash Equilibrium then the decomposable function cannot have one and at least one agent must be lying about their true preferences.

Maskin's work revealed a deeper truth: all incentive compatible mechanisms will induce the revelation of private information one way or the other, meaning that the difference between mechanisms is not whether they reveal user preferences so much as whether they reveal those preferences \textit{directly} or \textit{indirectly}. In direct mechanisms participants share private information truthfully in the pre-exchange negotiation step, allowing composable algorithms like auction mechanisms to calculate optimal allocation strategies and carry them out. In \textit{indirect} mechanisms participants reveal their preferences either obliquely in the price-discovery process (such as by preference-ranking bundles of goods) or by skipping the price-discovery stage and submitting purchase orders directly onto the market.

As a result, given that \textit{TFMs} are \textit{informationally decentralized} mechanisms in which users and producers make strategic decisions on the basis of private preferences over resource allocation, if our social choice rule is \textit{pareto optimality}, we cannot achieve it in any mechanism where participants can costlessly manipulate any information needed to estimate market pricing for transaction inclusion. If a mechanism permits block producers to costlessly include their own transactions in blocks we thus have \textit{de facto} grounds for concluding that incentive compatibility with \textit{pareto optimal} fee-throughput is impossible to sustain. Conflict motivated by \textit{strategic manipulation} can only be eliminated by making the inclusion of self-generated transactions costly, such that the decision by a block producer or user to use the blockchain reveals private information that the mechanism can exploit to shift overall provision into a more efficient equilibrium.

Hurwicz (1973) provides several other conditions any \textit{TFM} will need to meet in order to successfully implement \textit{pareto optimality}. The first is that one-shot mechanisms are insufficient, since algorithms with \textit{inertia} are required to iterate price levels into their optimal positions over time. This suggests that the information required to calculate the price of blockspace must be observable from the environment rather than collected from peers. And as Jordan (1986) observes, some form of smoothing of costs or payouts is beneficial to prevent mechanisms unpredictably oscillating around the desired equilibrium point.

In summary, our three types of goal conflict -- self-interest, free-riding, and strategic manipulation -- are distinct issues that affect most \textit{TFMs}. The first can only be eliminated building a mechanism that implements \textit{pareto optimality} as its social choice rule. But the second and third types are known in economics to prevent mechanisms from sustaining that equilibrium. All three forms of goal conflict thus induce existential attacks on the sustainability of the consensus mechanism. A threshold user who underbids in a Vickrey-Clarke-Groves auction is exhibiting self-interest. Users who conspire with producers to defund the security budget are free-riding on their non-colluding counterparts. A block producer who floods the network with spam transactions to drive up fees is engaging in strategic manipulation. 

A full solution requires eliminating all three kinds of goal conflict: killing free-riding pressures are as essential to security as imposing a cost on the publication of false market data.

With this economic framework in place, in the following section we turn our attention to the existing literature on \textit{TFMs} in computer science, with the goal of showing why the impossibility results in these papers reflect the limitations of their models and approaches rather than the limits of what is possible in distributed systems.

