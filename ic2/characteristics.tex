
The novel characteristics of \textit{TFMs} that lead to suboptimal provision are \emph{non-excludability}, \emph{self-provision} and \emph{informational decentralization}. 

\begin{enumerate}[leftmargin=*,itemsep=0.2em]
\item \textbf{Non-Excludability} allows anyone to use or provision these networks on equal terms provided they are willing to pay a competitive market price.
\item \textbf{Self-Provision} implies the existence of a payout that allows \textit{TFMs} to incentivize their own provision in the absence of an owner trusted third party.
\item \textbf{Informational Decentralization} refers to the informational property described by Hurwicz for mechanisms in which "participants have direction information only about themselves."
\end{enumerate}

The first characteristic of \textit{non-excludability} is an economic characteristic that provides for egalitarian usage of the network. As such, \textit{non-excludability} is a defining characteristic of \textit{TFMs} as it underpins the technical properties of \textit{censorship resistance}, \textit{decentralization} and \textit{network resilience}: censorship requires a mechanism with the power to exclude; centralization implies barriers to entry; resilience comes from the ability of participants to route around byzantine peers by adding their replacements to the network. Non-excludability also contributes to economic efficiency in \textit{TFMs}, as efficiency is maximized when producers build atop blocks proposed by their peers rather than orphaning them.

The second characteristic of \textit{informational decentralization} is not the casual concept of \textit{decentralization} as used in computer science and commonly-invoked to justify design decisions. It refers expliitly to the economic definition of \textit{informational decentralization} as used in mechanism design to describe mechanisms with the property of \textit{privacy} in which knowledge of individual resources and preferences is known only to those individuals. This characteristic makes \textit{TFMs} vulnerable to the attacks, identified by Hurwicz, in which participants manipulate the informational environment that others rely on to make strategic decisions.

The third characteristic of \textit{self-provision} implies the existence either of a fee-mechanism that collects fees from users and distributes them to network operators, or an inflationary block reward that pulls tokens into existence to compensate node operators for operating the network. While volunteer-run networks are theoretically possible, their designs fall outside the scope of \textit{TFMs} as transaction fees are purely redistributive. For this reason, in volunteer mechanisms the imposition of fees leads to a dead-weight efficiency loss, since any fee-level above zero is strictly sub-optimal given the cost structure of the network.

These three characteristics create fundamental tensions that \textit{TFMs} struggle to reconcile. Mechanisms must permit non-excludable access without enabling sybil attacks, incentivize provision with private benefits without socializing losses, and use decomposable algorithms that resist byzantine attacks on the message-passing layer. We can see the importance of all three characteristics from the way they form an \emph{economic trilemma} where the removal of any one property offers immediate relief to the problems created by the other two.

Understanding these characteristics allows us to identify the specific types of \textit{goal conflict} that motivate for-profit attacks on \textit{TFMs}. The first type, conflict rooted in \textit{self-interest}, occurs when participants prefer to allocate their resources differently than the mechanism designer intends. For an example of this, a user might desire to save a portion of their transaction fee to purchase a cheaper form of utility available elsewhere in the economy. In this case, participants are signalling disagreement with the designer's intended allocation of utility, either \textit{within} the mechanism or \textit{between} the mechanism and other external goods. These conflicts consequently involve participants choosing to bid at suboptimal fee levels, as they prioritize their personal preferences over the collective optimal outcome.

The second form of goal-conflict observable in \textit{TFMs} is \textit{free-riding}, which emerges when the characteristics of \textit{non-excludability} and \textit{self-provision} combine to create public goods within the consensus mechanism. While free-rider pressures are common in many mechanisms, in \textit{TFMs} they are particularly intractable due to the presence of two-sided free-rider problems where users and producers free-ride on the mechanism in different ways: producers by maximizing the revenue they extract from any collective payout like the block reward, and users by minimizing their contribution to the security budget. As our next section explains, these are the class of attacks that manifest in the form of side-contract payments, while is why attempts to mitigate other forms of \textit{goal conflict} fail to fully disincentivize collusion.

Our third form of goal-conflict is \textit{strategic manipulation}, which emerges because -- as Leonid Hurwicz observed -- in informationally-decentralized mechanisms participants can strategically manipulate others into suboptimally allocating their own resources by manipulating the informational space in which they conduct price-discovery and form rational strategies to optimize their spending. This type of goal-conflict is what incentivizes producers to create fake transactions and fake blocks, and what incentivizes users and producers to exploit threshold vulnerabilities in auction designs. This is the main problem mechanism designers have historically sought to eliminate through the use of techniques that attempt to achieve bayesian incentive compatibility or otherwise incentivize the truthful revelation of preferences.

As should be obvious, disagreements motivated by \textit{self-interest}, \textit{free-riding}, and \textit{strategic manipulation} are fundamentally distinct types of \textit{goal conflict}. The source of the first is psychological: in the private perception of the individual that their utility is better maximized through a different resource allocation strategy. The source of the second is inherent to the nature of the public  minimize indivisible benefits encourage participants to minimize their own contributions. And the source of the third is in the informational environment of the market itself, where the costless ability for participants to mislead others allows rational actors to induce others into forming strategies that misallocate their resources to the benefit the manipulating party.

The fact that conflicts motivated by \textit{self-interest}, \textit{free-riding} and \textit{strategic manipulation} constitute distinct types of goal-conflict is also why each type of attack within \textit{TMFs} is expressed in different ways and through unique social dynamics. Conflicts motivated by considerations of self-interest are expressed through unilateral changes to the fees offered for transaction inclusion by the fee-paying user. Conflicts motivated by an incentive to free-ride require cooperative attempts to defund the mechanism by a subset of network participants, since at least one producer must team up with at least one user in order to enable either to free-ride on the mechanism. Conflicts motivated by strategic manipulation are adversarial price-manipulation strategies such as bid-shading by users or the costless inclusion of transactions that manipulate fee expectations by producers.

The fundamentally different sources of these motivations for subverting \textit{TFM optimlity} is the primary reason achieving fee-optimality seems like an intractable problem. Any full solution requires the \textit{TFM} to implement an equilibrium in which none of these conflicts exist, which requires simultaneously eliminating unilateral, cooperative and adversarial strategies. It is little surprise that the existing literature has concluded this is an insurmountable task, especially in the face of methodological tools that are intended to address only the third problem. Yet a solution does exist, which is why our next section pulls back to economic theory to show why pareto optimality must be the social choice rule implemented by the solution.



