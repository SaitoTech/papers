
In the body of this paper, we demonstrated previous impossibility results simply universalize the limitations of their preferred \textit{direct mechanisms} and introduced an \textit{indirect mechanism} which is not subject to these limitations.

While the previous section shows the preceding impossibility results do not apply to \ourTFM, in order to conclusively illustrate that \ourTFM successfully overcomes the informational hurdles to implementing \textit{pareto optimality} we must return to economics and show how \ourTFM overcomes the foundational impediments to achieving \textit{pareto optimality} discussed in our preceding review of economic theory: Samuelson's objection based on the existence of public goods, and Hurwicz's objection on the ability for participants to costlessly subvert efficient price-discovery in the informationally decentralized environment.

\paragraph{Samuelson and Free-Riding}

The objectio that Samuelson offered to achieving \textit{pareto optimality} was based on the two-good equation for the \textit{utility possibilities frontier} which describes all points at which economic production is pareto optimal. His observation was that achieving this equation and thus pareto optimality is problematic in the presence of public goods with non-excludable forms of utility: rational actors will not keep the equation that balances utility production with its cost-of-production in equilibrium if they can enjoy the benefits of goods without the need to allocate resources in payment of the costs:

\LARGE
\begin{adjustwidth}{1.5em}{1.5em}
\begin{math}
\sum_{i=1}^{s} \frac{u_{{pub}+{priv}}^i}{u_b^i} = \frac{F_{{pub}+{priv}}}{F_b}
\end{math}
\end{adjustwidth}
\normalsize

To observe how \ourTFM solves this problem, note that transaction inclusion in our framework is neither a private good as conceptualized by Roughgarden nor a public good as conceptualized by Fox. The fee paid for blockspace is privately-collected and can be privately-negotiated, but collective security is maximized only to the extent its existence induces competition between producers for the right to collect the fee. This is why transaction hoarding strategies typically manifest in \textit{TFMs} with transaction fees: restricting the dissemination of transactions can limit the degree of competition for fee collection, and reduce the need for nodes to spend competitively on the security function.

\ourTFM skirts this problem through two approaches. The first involves the derivation of the work required to produce a block directly from the transaction fee itself. This eliminates the ability for producers to hold expected utility constant while offering a lower fee to users. Producers who offer participants discounted rates through off-chain payments must add their own fees back into blocks in a separate transaction in order to make up for the shortfall in routing work that results from any underpayment.

The second reason routing work eliminates \textit{free-riding} pressures comes from the explicit incentive it provides participants to broadcast transactions. We can see how this avoids the problem that Samuelson raised by modifying his cost function and adding a variable \textit{x} that reflects the probabiliy that transactions and fees are circulating publicly, such that open competition thus exists for collection of the transaction fee:

\LARGE
\begin{adjustwidth}{1.5em}{1.5em}
\begin{math}
\sum_{i=1}^{s} \frac{u_{({pub}*{x})+{priv}}^i}{u_b^i} = \frac{F_{{priv}}}{F_b}
\end{math}
\end{adjustwidth}
\normalsize

Theoretically, we know that users prefer widespread distribution of their fee as this maximizes the speed of transaction inclusion. And producers prefer to have private access to fees as this improves their relative profitability. Given the fact that routing work incentivizes producers to cooperatively share transactions, we can see that these mechanisms avoid the problems Samuelson flagged with suboptimality as the equation for the \textit{utility possibilities frontier} simplifies to the following once \textit{x} becomes 1:

\LARGE
\begin{adjustwidth}{1.5em}{1.5em}
\begin{math}
\sum_{i=1}^{s} \frac{u_{{pub}+{priv}}^i}{u_b^i} = \frac{F_{{priv}}}{F_b}
\end{math}
\end{adjustwidth}
\normalsize

Pareto optimality is achievable in this situation since \textit{free-riding pressures} are fully eliminated.

\paragraph{Hurwicz and the Incentive to Truthfulness}

The objection that Hurwicz offers to achieving incentive compatibility is based on the informational need for participants to engage in a process of price discovery priour to allocating resources or computing their own utility-maximizing strategies. This is the source of Hurwicz' distinction between the "pre-exchange negotiation stage" in which participants share information and the "action stage" in which they effect the resulting trades. Hurwicz argues that this distinction is always needed to achieve \textit{pareto optimality} as all algorithms capable of optimizing prices over time require users to form resource allocation strategies on the basis of a pre-computed understanding of the relative costs of different forms of utility.

As Hurwicz makes clear in his 1972 paper on this topic, it is consequently the lack of an "incentive to truthfulness" creates the potential for participants to engage in \textit{strategic manipulation}. Costless misrepresentation of the informational environment is what induces others to a suboptimal allocation of their own resources. When Roughgarden and his peers argue that costless transaction inclusion is a fundamental limitation of all \textit{TFMs}, they are offering a technologically-instantiated version of this critique, and a tautological assumption that makes it impossible to solve once incorporated into their equations.

The first way in which \ourTFM overcomes this issue is by moving the information needed to calculate prices out of the hands of adversarial peers and into what Hurwicz called the mechanism "environment". By listing the cost of transaction inclusion listed directly in the block header in the form of the burn fee needed to produce blocks, and with a wrap-around cost for chain-extension rooted in the real-world hash expenses needed to unlock payouts, calculating the market rate for transaction-inclusion becomes a mathematical exercise that can be performed without the need for off-chain price discovery. Participants can theoretically model the market price by examining the blockchain at whatever level historical granularity is needed for the purposes of price estimation.

But don't users get environmental information from their peers? While it might seem that block producers can forge the information as a form of strategic manipulation -- this is possible in many mechanisms -- \ourTFM offers a curious design that makes this quantifiably costly.

Abstractly, we can consider \textit{pareto optimality} as targeting an unknown price level which is most efficient at producing utility. We do not know this specific price level, but we know that we will reach this point if all transactions which are willing to pay the market rate are included, and no transactions which do not pay the market rate are included. The ability to create a mechanism that punishes both the exclusion of work funded by others and the inclusion of self-funded work thus creates a mechanism that imposes a cost on pushing prices away from their most efficient levels.

An asymmetrical cost is thus created that punishes \textit{strategic manipulation} by making the communication of fraudulant information costly -- imposed by regulating costs according to the measured efficiency of the topological channels through which the fees paid for broadcast flow -- thus overcoming Hurwicz' fundamental objection and permit algorithmic compatibility with pareto optimality. The mechanism removes all attacks motivated by \textit{strategic manipulation} by removing the ability for participants to manipulate price expectations.

On a closing note, we also observe that \ourTFM overcomes the other barriers to achieving \textit{pareto optimality} which are not discussed in computer science literature but nonetheless exist. The existence of a historical chain of blocks permits the user of price-discovery algorithms that require \textit{inertia} to achieve price optimality. And we note that the presence of algorithmic smoothing in both the cost and payout functions of the mechanisms adds for slight friction in the price-adjustment process that overcomes the objections of critics like Jordan (1986).

